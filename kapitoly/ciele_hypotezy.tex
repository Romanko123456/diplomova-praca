% ============================================================================
% CIELE A HYPOTÉZY
% ============================================================================

\chapter{Ciele a hypotézy práce}

\section{Ciele práce}

Cieľom diplomovej práce je na základe dostupnej literatúry a~vlastného výskumu posúdiť účinnosť fyzioterapie pri liečbe pacientov s~reumatoidnou artritídou.

\textbf{Hlavný cieľ:}

Zistiť, ako kombinovaná liečba interferenčnými prúdmi a~kinezioterapiou ovplyvňuje intenzitu bolesti, rozsah pohybu a~subjektívne hodnotenie kvality života pacientov s~reumatoidnou artritídou.

\textbf{Čiastkové ciele:}
\begin{enumerate}
    \item Porovnať účinnosť kombinovanej terapie (IFP + kinezioterapia) a~samotnej kinezioterapie na intenzitu bolesti.
    \item Analyzovať vplyv kombinovanej terapie na rozsah pohybu v~postihnutých kĺboch.
    \item Zhodnotiť subjektívne vnímanie kvality života pacientov pred a~po terapii.
    \item Porovnať výsledky experimentálnej a~kontrolnej skupiny.
\end{enumerate}

\section{Hypotézy}

Na základe stanovených cieľov sme formulovali nasledujúce hypotézy:

\textbf{H1:} Predpokladáme, že pacienti s~reumatoidnou artritídou, ktorí absolvujú terapiu interferenčnými prúdmi v~kombinácii s~kinezioterapiou, uvádzajú nižšiu intenzitu bolesti než pacienti liečení len kinezioterapiou.

\textbf{H2:} Predpokladáme, že kombinácia interferenčných prúdov a~kinezioterapie má väčší pozitívny vplyv na rozsah pohybu v~postihnutých kĺboch ako samotná kinezioterapia.

\textbf{H3:} Predpokladáme, že pacienti liečení kombinovanou terapiou subjektívne hodnotia kvalitu svojho života pozitívnejšie ako pacienti v~skupine bez elektroterapie.

\clearpage
