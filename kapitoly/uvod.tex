% ============================================================================
% ÚVOD
% Podľa metodických pokynov FZ TnUAD
% Zdôvodnenie aktuálnosti a významu témy
% Písané v 1. osobe množného čísla, minulý čas (autorský plurál)
% ============================================================================

\chapter*{Úvod}
\addcontentsline{toc}{chapter}{Úvod}

% TODO: Doplniť a rozšíriť úvod
Reumatoidná artritída (RA) patrí medzi najčastejšie chronické zápalové ochorenia pohybového aparátu, ktoré výrazne ovplyvňuje kvalitu života postihnutých pacientov. Toto autoimunitné ochorenie postihuje predovšetkým synoviálne kĺby a~bez adekvátnej liečby vedie k~progresívnej deštrukcii kĺbových štruktúr, bolesti a~funkčnému obmedzeniu. Podľa epidemiologických údajov postihuje RA približne 0,5--1~\% celosvetovej populácie, pričom ženy sú postihnuté trikrát častejšie ako muži.

Komplexná liečba reumatoidnej artritídy zahŕňa farmakoterapiu, chirurgickú liečbu a~rehabilitáciu. Fyzioterapia predstavuje neoddeliteľnú súčasť liečebného procesu, pričom jej cieľom je zmierniť bolesť, zachovať alebo zlepšiť pohyblivosť kĺbov a~udržať funkčnú zdatnosť pacienta. V~posledných rokoch sa do popredia dostávajú kombinované terapeutické prístupy, ktoré integrujú rôzne fyzikálne modality s~kinezioterapiou.

Medzi moderné fyzioterapeutické metódy patrí elektroterapia, konkrétne aplikácia interferenčných prúdov. Tieto strednofrekvenčné prúdy majú preukázaný analgetický, vazodilatačný a~myorelaxačný účinok. V~kombinácii s~kinezioterapiou, ktorá zahŕňa aktívne a~pasívne cvičenia, môže byť dosiahnutý synergický efekt v~liečbe pacientov s~RA.

Napriek rozšírenému používaniu fyzioterapeutických metód v~praxi stále chýbajú dostatočné dôkazy o~účinnosti kombinovaných terapeutických protokolov u~pacientov s~reumatoidnou artritídou. Táto skutočnosť nás motivovala k~realizácii výskumu, ktorého výsledky prezentujeme v~tejto diplomovej práci.

Cieľom našej diplomovej práce bolo preskúmať účinnosť kombinovanej liečby interferenčnými prúdmi a~kinezioterapiou u~pacientov s~reumatoidnou artritídou. Práca sa zameriava na hodnotenie zmien v~intenzite bolesti, rozsahu pohybu v~postihnutých kĺboch a~subjektívnom vnímaní kvality života pacientov.

Teoretická časť práce poskytuje prehľad súčasných poznatkov o~reumatoidnej artritíde, jej diagnostike a~možnostiach fyzioterapeutickej liečby na základe analýzy domácej a~zahraničnej literatúry. Praktická časť obsahuje metodiku výskumu a~výsledky štúdie realizovanej na vzorke pacientov s~potvrdenou diagnózou RA.

\clearpage
