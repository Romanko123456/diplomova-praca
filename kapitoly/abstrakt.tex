% ============================================================================
% ABSTRAKT
% ============================================================================

\chapter*{Abstrakt}
\addcontentsline{toc}{chapter}{Abstrakt}

% Slovenský abstrakt
\textbf{BALLEK, Roman}: \textit{\nazovprace}. [Diplomová práca]. \univerzita. \fakulta; Vedúci práce: \veduci. Trenčín: TnUNI, \rok.

\vspace{0.5cm}

% TODO: Doplniť abstrakt v slovenčine
Diplomová práca sa zaoberá využitím fyzioterapie u~pacientov s~reumatoidnou artritídou. Hlavným cieľom práce je zistiť účinnosť kombinovanej liečby interferenčnými prúdmi a~kinezioterapiou v~porovnaní so samotnou kinezioterapiou. Práca sleduje vplyv terapie na intenzitu bolesti, rozsah pohybu v~postihnutých kĺboch a~subjektívne hodnotenie kvality života pacientov.

Teoretická časť práce definuje reumatoidnú artritídu, jej etiológiu, patogenézu, klinický obraz a~diagnostiku. Ďalej sa venuje možnostiam fyzioterapeutickej liečby so zameraním na interferenčné prúdy a~kinezioterapiu.

Praktická časť obsahuje výsledky výskumu realizovaného na vzorke pacientov s~potvrdenou diagnózou reumatoidnej artritídy. Pacienti boli rozdelení do experimentálnej a~kontrolnej skupiny. Údaje boli zbierané pomocou dotazníka vlastnej konštrukcie pred začiatkom a~po ukončení terapeutického cyklu.

\vspace{0.5cm}

\textbf{Kľúčové slová:} reumatoidná artritída, fyzioterapia, interferenčné prúdy, kinezioterapia, bolesť, kvalita života

\clearpage

% Anglický abstrakt
\chapter*{Abstract}
\addcontentsline{toc}{chapter}{Abstract}

\textbf{BALLEK, Roman}: \textit{\nazovpraceEN}. [Diploma thesis]. Alexander Dubček University of Trenčín. Faculty of Health Care; Supervisor: \veduci. Trenčín: TnUNI, \rok.

\vspace{0.5cm}

% TODO: Doplniť abstrakt v angličtine
The diploma thesis deals with the use of physiotherapy in patients with rheumatoid arthritis. The main objective of the thesis is to determine the effectiveness of combined treatment with interferential currents and kinesiotherapy compared to kinesiotherapy alone. The thesis monitors the effect of therapy on pain intensity, range of motion in affected joints, and subjective assessment of patients' quality of life.

The theoretical part of the thesis defines rheumatoid arthritis, its etiology, pathogenesis, clinical presentation, and diagnosis. It also addresses physiotherapeutic treatment options with a~focus on interferential currents and kinesiotherapy.

The practical part contains the results of research conducted on a~sample of patients with a~confirmed diagnosis of rheumatoid arthritis. Patients were divided into experimental and control groups. Data were collected using a~self-constructed questionnaire before the start and after the completion of the therapeutic cycle.

\vspace{0.5cm}

\textbf{Keywords:} rheumatoid arthritis, physiotherapy, interferential currents, kinesiotherapy, pain, quality of life

\clearpage
