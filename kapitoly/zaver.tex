% ============================================================================
% ZÁVER
% Podľa metodických pokynov FZ TnUAD
% Písané v 1. osobe množného čísla, minulý čas (autorský plurál)
% ============================================================================

\chapter*{Záver}
\addcontentsline{toc}{chapter}{Záver}

Diplomová práca sa zaoberala využitím fyzioterapie u~pacientov s~reumatoidnou artritídou so zameraním na hodnotenie účinnosti kombinovanej liečby interferenčnými prúdmi a~kinezioterapiou.

V~teoretickej časti práce sme na základe analýzy domácej a~zahraničnej literatúry definovali reumatoidnú artritídu ako chronické autoimunitné zápalové ochorenie postihujúce synoviálne kĺby. Predstavili sme súčasné poznatky o~etiológii, patogenéze, klinickom obraze a~diagnostike tohto ochorenia. Podrobne sme sa venovali možnostiam fyzioterapeutickej liečby so zameraním na interferenčné prúdy a~kinezioterapiu.

% TODO: Doplniť po ukončení výskumu - zhrnutie hlavných zistení

\textbf{K~hlavnému cieľu:}

Hlavným cieľom práce bolo zistiť, ako kombinovaná liečba interferenčnými prúdmi a~kinezioterapiou ovplyvňuje intenzitu bolesti, rozsah pohybu a~subjektívne hodnotenie kvality života pacientov s~reumatoidnou artritídou.

% [Doplniť zhrnutie hlavných zistení]

\textbf{K~hypotéze H1:}

Predpokladali sme, že pacienti s~RA, ktorí absolvujú terapiu interferenčnými prúdmi v~kombinácii s~kinezioterapiou, uvádzajú nižšiu intenzitu bolesti než pacienti liečení len kinezioterapiou.

% [Doplniť: Hypotéza bola/nebola potvrdená...]

\textbf{K~hypotéze H2:}

Predpokladali sme, že kombinácia interferenčných prúdov a~kinezioterapie má väčší pozitívny vplyv na rozsah pohybu v~postihnutých kĺboch ako samotná kinezioterapia.

% [Doplniť: Hypotéza bola/nebola potvrdená...]

\textbf{K~hypotéze H3:}

Predpokladali sme, že pacienti liečení kombinovanou terapiou subjektívne hodnotia kvalitu svojho života pozitívnejšie ako pacienti v~skupine bez elektroterapie.

% [Doplniť: Hypotéza bola/nebola potvrdená...]

\vspace{1cm}

Výsledky našej štúdie poukazujú na potenciálny prínos kombinovanej terapie v~rehabilitácii pacientov s~reumatoidnou artritídou. Fyzioterapia predstavuje neoddeliteľnú súčasť komplexnej liečby RA a~má nezastupiteľnú úlohu v~zlepšovaní funkčného stavu a~kvality života pacientov.

Pre klinickú prax odporúčame zaradenie interferenčných prúdov do terapeutického protokolu ako prípravnú fázu pred kinezioterapiou. Tento prístup môže zvýšiť účinnosť rehabilitácie a~zlepšiť toleranciu pohybovej liečby u~pacientov s~bolestivými stavmi.

Ďalší výskum by mal byť zameraný na dlhodobé sledovanie účinnosti kombinovanej terapie a~na optimalizáciu parametrov aplikácie interferenčných prúdov u~pacientov s~reumatickými ochoreniami.

\clearpage
