% ============================================================================
% DISKUSIA
% Podľa metodických pokynov FZ TnUAD
% Písané v 1. osobe množného čísla, minulý čas (autorský plurál)
% ============================================================================

\chapter{Diskusia}

V~tejto kapitole interpretujeme získané výsledky v~kontexte stanovených hypotéz a~porovnávame ich s~výsledkami podobných štúdií publikovaných v~odbornej literatúre.

\section{Interpretácia výsledkov}

% TODO: Doplniť interpretáciu po ukončení výskumu

\subsection{Vplyv kombinovanej terapie na intenzitu bolesti (H1)}

% TODO: Doplniť konkrétne výsledky

Naše zistenia sú v~súlade s~výsledkami systematického prehľadu Fuentesa a~kol. \parencite*{fuentes2010}, ktorí potvrdili účinnosť interferenčných prúdov v~redukcii muskuloskeletálnej bolesti. Rampazo a~Liebano \parencite*{Rampazo2022} vo svojom naratívnom prehľade vysvetľujú mechanizmy analgetického účinku IFP vrátane aktivácie endogénných opioidných mechanizmov a~teórie vrátkovej kontroly bolesti.

Kawa, Muszynska a~Kowza-Dzwonkowska \parencite*{Kawa2014} v~štúdii u~pacientov s~degeneratívnymi ochoreniami takisto pozorovali signifikantné zníženie bolesti po aplikácii interferenčných prúdov.

\subsection{Vplyv kombinovanej terapie na rozsah pohybu (H2)}

% TODO: Doplniť konkrétne výsledky

Zlepšenie rozsahu pohybu môže byť vysvetlené kombináciou niekoľkých faktorov. Interferenčné prúdy prispievajú k~uvoľneniu svalového napätia a~zmierneniu bolesti, čo umožňuje pacientom aktívnejšie sa zapájať do kinezioterapie \parencite{Watson2020}. Kinezioterapia následne vedie k~zlepšeniu pohyblivosti kĺbov prostredníctvom mobilizačných a~strečingových techník \parencite{Kolar2012}.

Hurkmans a~kol. \parencite*{hurkmans2009} v~Cochrane prehľade potvrdili, že dynamické cvičebné programy sú bezpečné a~účinné u~pacientov s~RA.

\subsection{Vplyv kombinovanej terapie na kvalitu života (H3)}

% TODO: Doplniť konkrétne výsledky

Krchňavá a~kol. \parencite*{Krchnava2018} vo svojej štúdii identifikovali bolesť ako jeden z~najvýznamnejších faktorov ovplyvňujúcich kvalitu života pacientov s~RA. Zníženie bolesti v~dôsledku kombinovanej terapie by teda malo viesť k~zlepšeniu subjektívneho hodnotenia kvality života.

Metsios a~Kitas \parencite*{metsios2020} zdôrazňujú, že pravidelná pohybová aktivita má pozitívny vplyv nielen na fyzické, ale aj na psychické zdravie pacientov s~RA.

\section{Porovnanie s~inými štúdiami}

% TODO: Porovnať výsledky s literatúrou

Naše výsledky môžeme porovnať s~niekoľkými relevantnými štúdiami. Zhang a~kol. \parencite*{Zhang2025} v~recentnej network meta-analýze hodnotili účinnosť rôznych cvičebných intervencií u~pacientov s~RA. Zistili, že kombinované intervencie majú väčší efekt než jednotlivé modality samostatne, čo podporuje náš kombinovaný prístup.

Gravaldi a~kol. \parencite*{Gravaldi2022} v~systematickom prehľade účinnosti fyzioterapie pri ankylozujúcej spondylitíde (inom zápalovom reumatickom ochorení) takisto potvrdili superióritu kombinovaných prístupov.

Herbert a~kol. \parencite*{Herbert2022} v~príručke Evidence-Based Physiotherapy zdôrazňujú dôležitosť hodnotenia účinnosti na základe zmysluplných klinických rozdielov, nielen štatistickej významnosti.

\section{Limity štúdie}

Pri interpretácii výsledkov je potrebné zohľadniť limity našej štúdie:

\begin{itemize}
    \item \textbf{Veľkosť výskumného súboru} -- počet respondentov mohol byť nedostatočný pre detekciu menších rozdielov medzi skupinami
    
    \item \textbf{Subjektívnosť niektorých hodnotených parametrov} -- hodnotenie bolesti pomocou VAS a~subjektívne vnímanie kvality života sú závislé od individuálneho vnímania pacienta
    
    \item \textbf{Nemožnosť zaslepenia} -- vzhľadom na povahu intervencie nebolo možné zaslepiť pacientov ani terapeutov, čo mohlo viesť k~placebovému efektu
    
    \item \textbf{Heterogenita súboru} -- pacienti mohli mať rôzny stupeň aktivity ochorenia a~rôznu dĺžku trvania RA
    
    \item \textbf{Krátkodobé sledovanie} -- hodnotili sme bezprostredný efekt terapie, dlhodobý účinok nebol sledovaný
\end{itemize}

\section{Silné stránky štúdie}

\begin{itemize}
    \item Porovnanie výsledkov s~kontrolnou skupinou
    \item Hodnotenie viacerých parametrov (bolesť, ROM, kvalita života)
    \item Použitie validovaných nástrojov na hodnotenie výsledkov
    \item Štandardizovaný terapeutický protokol
\end{itemize}

\section{Odporúčania pre prax}

Na základe získaných výsledkov formulujeme nasledujúce odporúčania pre klinickú prax:

\begin{enumerate}
    \item Kombinácia interferenčných prúdov s~kinezioterapiou môže byť efektívnym prístupom v~rehabilitácii pacientov s~RA
    
    \item Aplikácia IFP pred cvičením môže zlepšiť toleranciu pohybovej liečby u~pacientov s~bolesťou
    
    \item Je potrebné individualizovať terapeutický protokol podľa aktuálneho stavu pacienta
    
    \item Pacienti by mali byť edukovaní o~dôležitosti pravidelnej pohybovej aktivity
\end{enumerate}

\section{Odporúčania pre ďalší výskum}

\begin{itemize}
    \item Realizácia randomizovanej kontrolovanej štúdie s~väčším súborom
    \item Dlhodobé sledovanie účinnosti (follow-up)
    \item Porovnanie rôznych parametrov aplikácie IFP
    \item Hodnotenie objektívnych parametrov (zápalové markery, zobrazovacie metódy)
\end{itemize}

\clearpage
