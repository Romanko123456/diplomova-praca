% ============================================================================
% METODIKA
% ============================================================================

\chapter{Metodika práce}

\section{Charakteristika výskumného súboru}

Výskumný súbor tvorili pacienti s~potvrdenou diagnózou reumatoidnej artritídy. Podmienkou zaradenia do štúdie bola potvrdená diagnóza RA lekárom.

Predpokladaný počet respondentov: 100

Pacienti boli rozdelení do dvoch skupín:
\begin{itemize}
    \item \textbf{Experimentálna skupina} -- pacienti absolvujúci kombinovanú liečbu interferenčným prúdom a~kinezioterapiou
    \item \textbf{Kontrolná skupina} -- pacienti podstupujúci iba kinezioterapiu
\end{itemize}

\section{Metódy zberu dát}

Na zber údajov bol vytvorený dotazník vlastnej konštrukcie, ktorý pacienti vypĺňali priamo v~ambulancii pred začiatkom rehabilitačného cyklu a~po jeho ukončení.

Pred začiatkom liečby boli pacienti poučení o~správnom vyplnení dotazníkov, účele ich využitia a~o~anonymite osobných údajov.

Zber dát prebiehal od augusta 2025 do februára 2026.

\subsection{Štruktúra dotazníka}

Dotazník obsahoval:
\begin{itemize}
    \item \textbf{Demografické údaje} -- pomáhajú charakterizovať skúmanú vzorku (vek, pohlavie, dĺžka trvania ochorenia)
    \item \textbf{Intenzita bolesti} -- hodnotená pomocou VAS škály
    \item \textbf{Pohyblivosť kĺbov} -- subjektívne hodnotenie
    \item \textbf{Kvalita spánku} -- vplyv bolesti na spánok
    \item \textbf{Celková subjektívna kvalita života} -- hodnotenie pred a~po terapeutickom cykle
\end{itemize}

\section{Terapeutický protokol}

\subsection{Experimentálna skupina}

Pacienti v~experimentálnej skupine absolvovali:
\begin{enumerate}
    \item Aplikáciu interferenčných prúdov na postihnuté kĺby
    \item Kinezioterapiu zameranú na:
    \begin{itemize}
        \item Udržanie a~zlepšenie rozsahu pohybu
        \item Posilnenie svalstva
        \item Nácvik jemnej motoriky
    \end{itemize}
\end{enumerate}

\subsection{Kontrolná skupina}

Pacienti v~kontrolnej skupine absolvovali iba kinezioterapiu bez aplikácie elektroterapie.

\section{Metódy spracovania a analýzy dát}

Získané údaje boli spracované formou grafov:
\begin{itemize}
    \item \textbf{Stĺpcové grafy} -- pre jednoduchšie otázky typu áno/nie alebo kategorizáciu bolesti
    \item \textbf{Krabicové grafy} -- pre porovnanie intenzity bolesti, rozsahu pohybu a~hodnotenia kvality života pred a~po terapii
\end{itemize}

\subsection{Deskriptívna štatistika}

Pre základnú charakterizáciu vzorky bola využitá deskriptívna štatistika:
\begin{itemize}
    \item Aritmetický priemer
    \item Smerodajná odchýlka
    \item Minimum a~maximum
    \item Absolútne hodnoty
\end{itemize}

\subsection{Inferenčná štatistika}

Na vyhodnotenie štatistickej významnosti zmien bol použitý neparametrický \textbf{Wilcoxonov párový test}, vzhľadom na predpokladané nenormálne rozdelenie údajov.

Tento postup umožnil porovnať účinnosť kombinovanej liečby interferenčnými prúdmi a~kinezioterapiou s~izolovanou kinezioterapiou a~sledovať zmeny v:
\begin{itemize}
    \item Bolesti
    \item Pohyblivosti
    \item Subjektívnom hodnotení kvality života
\end{itemize}

\section{Etické aspekty výskumu}

Všetci pacienti boli informovaní o~účele výskumu a~dobrovoľne súhlasili s~účasťou. Zber dát bol anonymný a~údaje boli použité výlučne na výskumné účely.

\clearpage
