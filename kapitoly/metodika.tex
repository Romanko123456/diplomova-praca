% ============================================================================
% METODIKA
% Podľa metodických pokynov FZ TnUAD
% Písané v 1. osobe množného čísla, minulý čas (autorský plurál)
% ============================================================================

\chapter{Cieľ práce a metodológia}

\section{Cieľ práce}

Cieľom diplomovej práce bolo na základe dostupnej literatúry a~vlastného výskumu posúdiť účinnosť fyzioterapie pri liečbe pacientov s~reumatoidnou artritídou.

\textbf{Hlavný cieľ:}
Zistiť, ako kombinovaná liečba interferenčnými prúdmi a~kinezioterapiou ovplyvňuje intenzitu bolesti, rozsah pohybu a~subjektívne hodnotenie kvality života pacientov s~reumatoidnou artritídou.

\textbf{Čiastkové ciele:}
\begin{enumerate}
    \item Porovnať účinnosť kombinovanej terapie (IFP + kinezioterapia) a~samotnej kinezioterapie na intenzitu bolesti.
    \item Analyzovať vplyv kombinovanej terapie na rozsah pohybu v~postihnutých kĺboch.
    \item Zhodnotiť subjektívne vnímanie kvality života pacientov pred a~po terapii.
    \item Porovnať výsledky experimentálnej a~kontrolnej skupiny.
\end{enumerate}

\section{Hypotézy}

Na základe stanovených cieľov sme formulovali nasledujúce hypotézy:

\textbf{H1:} Predpokladali sme, že pacienti s~reumatoidnou artritídou, ktorí absolvujú terapiu interferenčnými prúdmi v~kombinácii s~kinezioterapiou, uvádzajú nižšiu intenzitu bolesti než pacienti liečení len kinezioterapiou.

\textbf{H2:} Predpokladali sme, že kombinácia interferenčných prúdov a~kinezioterapie má väčší pozitívny vplyv na rozsah pohybu v~postihnutých kĺboch ako samotná kinezioterapia.

\textbf{H3:} Predpokladali sme, že pacienti liečení kombinovanou terapiou subjektívne hodnotia kvalitu svojho života pozitívnejšie ako pacienti v~skupine bez elektroterapie.

\section{Objekt a predmet výskumu}

\textbf{Objekt výskumu:} Pacienti s~potvrdenou diagnózou reumatoidnej artritídy.

\textbf{Predmet výskumu:} Účinnosť kombinovanej fyzioterapeutickej liečby (interferenčné prúdy + kinezioterapia) na intenzitu bolesti, rozsah pohybu a~kvalitu života.

\section{Charakteristika výskumného súboru}

Výskumný súbor tvorili pacienti s~potvrdenou diagnózou reumatoidnej artritídy. Podmienkou zaradenia do štúdie bola lekárom potvrdená diagnóza RA.

Predpokladaný počet respondentov: 100

Pacientov sme rozdelili do dvoch skupín:
\begin{itemize}
    \item \textbf{Experimentálna skupina} -- pacienti absolvujúci kombinovanú liečbu interferenčným prúdom a~kinezioterapiou
    \item \textbf{Kontrolná skupina} -- pacienti podstupujúci iba kinezioterapiu
\end{itemize}

\section{Metódy zberu dát}

Na zber údajov sme vytvorili dotazník vlastnej konštrukcie, ktorý pacienti vypĺňali priamo v~ambulancii pred začiatkom rehabilitačného cyklu a~po jeho ukončení.

Pred začiatkom liečby sme pacientov poučili o~správnom vyplnení dotazníkov, účele ich využitia a~o~anonymite osobných údajov.

Zber dát prebiehal od augusta 2025 do februára 2026.

\subsection{Štruktúra dotazníka}

Dotazník obsahoval:
\begin{itemize}
    \item \textbf{Demografické údaje} -- pomáhali charakterizovať skúmanú vzorku (vek, pohlavie, dĺžka trvania ochorenia)
    \item \textbf{Intenzita bolesti} -- hodnotená pomocou VAS škály
    \item \textbf{Pohyblivosť kĺbov} -- subjektívne hodnotenie
    \item \textbf{Kvalita spánku} -- vplyv bolesti na spánok
    \item \textbf{Celková subjektívna kvalita života} -- hodnotenie pred a~po terapeutickom cykle
\end{itemize}

\section{Terapeutický protokol}

\subsection{Experimentálna skupina}

Pacienti v~experimentálnej skupine absolvovali:
\begin{enumerate}
    \item Aplikáciu interferenčných prúdov na postihnuté kĺby
    \item Kinezioterapiu zameranú na:
    \begin{itemize}
        \item Udržanie a~zlepšenie rozsahu pohybu
        \item Posilnenie svalstva
        \item Nácvik jemnej motoriky
    \end{itemize}
\end{enumerate}

\subsection{Kontrolná skupina}

Pacienti v~kontrolnej skupine absolvovali iba kinezioterapiu bez aplikácie elektroterapie.

\section{Metódy spracovania a analýzy dát}

Získané údaje sme spracovali formou grafov:
\begin{itemize}
    \item \textbf{Stĺpcové grafy} -- pre jednoduchšie otázky typu áno/nie alebo kategorizáciu bolesti
    \item \textbf{Krabicové grafy} -- pre porovnanie intenzity bolesti, rozsahu pohybu a~hodnotenia kvality života pred a~po terapii
\end{itemize}

\subsection{Deskriptívna štatistika}

Pre základnú charakterizáciu vzorky sme využili deskriptívnu štatistiku:
\begin{itemize}
    \item Aritmetický priemer
    \item Smerodajná odchýlka
    \item Minimum a~maximum
    \item Absolútne hodnoty
\end{itemize}

\subsection{Inferenčná štatistika}

Na vyhodnotenie štatistickej významnosti zmien sme použili neparametrický \textbf{Wilcoxonov párový test}, vzhľadom na predpokladané nenormálne rozdelenie údajov.

Tento postup nám umožnil porovnať účinnosť kombinovanej liečby interferenčnými prúdmi a~kinezioterapiou s~izolovanou kinezioterapiou a~sledovať zmeny v:
\begin{itemize}
    \item Bolesti
    \item Pohyblivosti
    \item Subjektívnom hodnotení kvality života
\end{itemize}

\section{Etické aspekty výskumu}

Všetkých pacientov sme informovali o~účele výskumu a~dobrovoľne súhlasili s~účasťou. Zber dát bol anonymný a~údaje boli použité výlučne na výskumné účely.

\clearpage
