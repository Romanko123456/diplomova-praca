% ============================================================================
% TEORETICKÁ ČASŤ
% ============================================================================

\chapter{Teoretická časť}

\section{Reumatoidná artritída}

\subsection{Definícia a epidemiológia}

% TODO: Doplniť obsah
Reumatoidná artritída (RA) je chronické systémové autoimunitné zápalové ochorenie, ktoré primárne postihuje synoviálne kĺby. Charakterizuje sa symetrickou polyartritídou s~predilekciou malých kĺbov rúk a~nôh, pričom bez adekvátnej liečby vedie k~progresívnej deštrukcii kĺbových štruktúr.

Prevalencia RA sa pohybuje okolo 0,5--1~\% celosvetovej populácie. Ochorenie postihuje častejšie ženy než mužov v~pomere približne 3:1. Vrchol incidencie sa nachádza medzi 40.--60. rokom života, avšak RA sa môže manifestovať v~akomkoľvek veku.

\subsection{Etiológia a patogenéza}

% TODO: Doplniť obsah
Etiológia RA nie je úplne objasnená, avšak predpokladá sa multifaktoriálny pôvod zahŕňajúci genetickú predispozíciu, environmentálne faktory a~imunologickú dysreguláciu.

Genetické faktory zohrávajú významnú úlohu v~rozvoji ochorenia. Najviac je asociovaná prítomnosť HLA-DR4 a~HLA-DR1 antigénov. Environmentálne faktory zahŕňajú fajčenie, infekčné agens a~hormonálne vplyvy.

Patogenéza RA je charakterizovaná chronickým zápalom synoviálnej membrány. Aktivované T-lymfocyty, makrofágy a~fibroblasty produkujú prozápalové cytokíny (TNF-$\alpha$, IL-1, IL-6), ktoré vedú k~proliferácii synovie a~tvorbe pannu. Panus invaduje chrupavku a~subchondrálnu kosť, čo vedie k~eróziám a~deštrukcii kĺbu.

\subsection{Klinický obraz}

% TODO: Doplniť obsah
Klinický obraz RA je variabilný. Typický začiatok je postupný, s~rannou stuhnutosťou trvajúcou viac ako 30 minút, bolesťou a~opuchom kĺbov.

Najčastejšie postihnuté kĺby sú:
\begin{itemize}
    \item Metakarpofalangeálne (MCP) kĺby
    \item Proximálne interfalangeálne (PIP) kĺby
    \item Zápästia
    \item Metatarzofalangeálne (MTP) kĺby
\end{itemize}

Mimokĺbové prejavy zahŕňajú reumatoidné uzlíky, vaskulitídu, pľúcne postihnutie, anémiu a~sekundárny Sjögrenov syndróm.

\subsection{Diagnostika}

% TODO: Doplniť obsah
Diagnostika RA je založená na kombinácii klinických, laboratórnych a~zobrazovacích vyšetrení. Klasifikačné kritériá ACR/EULAR 2010 umožňujú včasnú diagnostiku ochorenia.

Laboratórne nálezy zahŕňajú:
\begin{itemize}
    \item Pozitivitu reumatoidného faktora (RF)
    \item Pozitivitu anti-CCP protilátok
    \item Zvýšené zápalové parametre (CRP, ESR)
\end{itemize}

Zobrazovacie metódy zahŕňajú:
\begin{itemize}
    \item Röntgenové vyšetrenie -- periartikulárna osteoporóza, erózie, zúženie kĺbovej štrbiny
    \item Ultrasonografia -- detekcia synovitídy, erózie
    \item Magnetická rezonancia -- včasná detekcia zápalových zmien
\end{itemize}

\section{Fyzioterapia pri reumatoidnej artritíde}

\subsection{Ciele fyzioterapie}

% TODO: Doplniť obsah
Fyzioterapia predstavuje integrálnu súčasť komplexnej liečby RA. Hlavné ciele zahŕňajú:
\begin{itemize}
    \item Zmiernenie bolesti
    \item Udržanie alebo zlepšenie rozsahu pohybu v~kĺboch
    \item Prevencia svalovej atrofie
    \item Zlepšenie funkčnej zdatnosti
    \item Edukácia pacienta o~ochrane kĺbov
    \item Zlepšenie kvality života
\end{itemize}

\subsection{Kinezioterapia}

% TODO: Doplniť obsah
Kinezioterapia je základnou súčasťou rehabilitácie pacientov s~RA. Zahŕňa pasívne, aktívne a~odporové cvičenia zamerané na udržanie pohyblivosti a~svalovej sily.

V akútnej fáze ochorenia sú indikované pasívne pohyby a~polohovanie. V~subakútnej a~chronickej fáze sa postupne zavádzajú aktívne cvičenia a~posilňovanie.

Hydrokinezioterapia využíva vlastnosti vodného prostredia -- vztlak, odpor a~teplo -- na uľahčenie pohybu a~zmiernenie bolesti.

\subsection{Elektroterapia}

% TODO: Doplniť obsah
Elektroterapia zahŕňa rôzne formy aplikácie elektrického prúdu na terapeutické účely. Pri RA sa využívajú predovšetkým:
\begin{itemize}
    \item TENS (transkutánna elektrická nervová stimulácia)
    \item Interferenčné prúdy
    \item Diadynamické prúdy
    \item Galvanoterapia
\end{itemize}

\subsection{Interferenčné prúdy}

% TODO: Doplniť obsah
Interferenčné prúdy (IFP) patria medzi strednefrekvenčnú elektroterapiu. Princíp spočíva v~aplikácii dvoch strednefrekvenčných prúdov (zvyčajne 4000--4100~Hz), ktoré sa v~tkanive interferujú a~vytvárajú nízkofrekvenčný modulovaný prúd.

Účinky interferenčných prúdov:
\begin{itemize}
    \item \textbf{Analgetický účinok} -- blokáda bolestivých vzruchov podľa teórie vrátkovej kontroly
    \item \textbf{Vazodilatačný účinok} -- zlepšenie prekrvenia tkanív
    \item \textbf{Myorelaxačný účinok} -- uvoľnenie svalového napätia
    \item \textbf{Trofotropný účinok} -- podpora regenerácie tkanív
\end{itemize}

Výhodou IFP je hlboký prienik do tkanív bez nepríjemných pocitov na povrchu kože, ktoré sú typické pre nízkofrekvenčné prúdy.

Kontraindikácie aplikácie IFP zahŕňajú:
\begin{itemize}
    \item Kardiostimulátor
    \item Malignity
    \item Akútne zápaly
    \item Tehotenstvo
    \item Porušenú integritu kože v~mieste aplikácie
\end{itemize}

\section{Hodnotenie účinnosti terapie}

\subsection{Hodnotenie bolesti}

% TODO: Doplniť obsah
Na hodnotenie intenzity bolesti sa najčastejšie používa vizuálna analógová škála (VAS). Pacient označí intenzitu bolesti na 10~cm úsečke, kde 0 predstavuje žiadnu bolesť a~10 najhoršiu predstaviteľnú bolesť.

\subsection{Hodnotenie rozsahu pohybu}

% TODO: Doplniť obsah
Rozsah pohybu (ROM -- Range of Motion) sa meria goniometrom. Hodnotí sa aktívny a~pasívny rozsah pohybu v~postihnutých kĺboch a~porovnáva sa s~fyziologickými hodnotami.

\subsection{Hodnotenie kvality života}

% TODO: Doplniť obsah
Na hodnotenie kvality života sa používajú štandardizované dotazníky ako SF-36 (Short Form 36 Health Survey) alebo HAQ (Health Assessment Questionnaire).

\clearpage
