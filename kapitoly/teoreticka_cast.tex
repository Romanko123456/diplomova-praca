% ============================================================================
% TEORETICKÁ ČASŤ
% Podľa metodických pokynov FZ TnUAD - cca 30% práce
% Písané v 1. osobe množného čísla, minulý čas (autorský plurál)
% ============================================================================

\chapter{Súčasný stav riešenej problematiky}

\section{Reumatoidná artritída}

\subsection{Definícia a epidemiológia}

Reumatoidná artritída (RA) je chronické systémové autoimunitné zápalové ochorenie, ktoré primárne postihuje synoviálne kĺby. Charakterizuje sa symetrickou polyartritídou s~predilekciou malých kĺbov rúk a~nôh, pričom bez adekvátnej liečby vedie k~progresívnej deštrukcii kĺbových štruktúr \parencite{smolen2018}. Podľa Macejovej \parencite*{Macejova2019} predstavuje RA najčastejšie zápalové kĺbové ochorenie v~našej populácii.

Prevalencia RA sa pohybuje okolo 0,5--1~\% celosvetovej populácie. Ochorenie postihuje častejšie ženy než mužov v~pomere približne 3:1. Vrchol incidencie sa nachádza medzi 40.--60. rokom života, avšak RA sa môže manifestovať v~akomkoľvek veku \parencite{firestein2021}. Olejárová a~kol. \parencite*{Olejarova2016} upozorňujú, že včasná diagnostika a~liečba sú kľúčové pre prevenciu ireverzibilných kĺbových zmien.

\subsection{Etiológia a patogenéza}

Etiológia RA nie je úplne objasnená, avšak predpokladá sa multifaktoriálny pôvod zahŕňajúci genetickú predispozíciu, environmentálne faktory a~imunologickú dysreguláciu \parencite{smolen2018}.

\subsubsection{Genetické faktory}

Genetické faktory zohrávajú významnú úlohu v~rozvoji ochorenia. Najviac je asociovaná prítomnosť HLA-DR4 a~HLA-DR1 antigénov, ktoré sa nachádzajú u~60--70~\% pacientov s~RA \parencite{firestein2021}. Rovenský, Kozák a~Štvrtinová \parencite*{Rovensky2015} uvádzajú, že genetická predispozícia vysvetľuje približne 50--60~\% rizika vzniku ochorenia.

\subsubsection{Environmentálne faktory}

Medzi environmentálne faktory patria predovšetkým:
\begin{itemize}
    \item \textbf{Fajčenie} -- najvýznamnejší modifikovateľný rizikový faktor
    \item \textbf{Infekčné agens} -- predpokladá sa úloha niektorých vírusov a~baktérií
    \item \textbf{Hormonálne vplyvy} -- vysvetľujú vyššiu prevalenciu u~žien
    \item \textbf{Mikrobiom} -- aktuálne skúmaná oblasť \parencite{smolen2018}
\end{itemize}

\subsubsection{Patogenéza}

Patogenéza RA je charakterizovaná chronickým zápalom synoviálnej membrány. Aktivované T-lymfocyty, makrofágy a~fibroblasty produkujú prozápalové cytokíny (TNF-$\alpha$, IL-1, IL-6), ktoré vedú k~proliferácii synovie a~tvorbe pannu \parencite{Macejova2019}. Panus invaduje chrupavku a~subchondrálnu kosť, čo vedie k~eróziám a~deštrukcii kĺbu. Tento proces je detailne popísaný vo viacerých odborných publikáciách \parencite{pavelka2018, Olejarova2016}.

\subsection{Klinický obraz}

Klinický obraz RA je variabilný. Typický začiatok je postupný, s~rannou stuhnutosťou trvajúcou viac ako 30 minút, bolesťou a~opuchom kĺbov \parencite{Macejova2019}.

\subsubsection{Kĺbové prejavy}

Najčastejšie postihnuté kĺby sú podľa Olejárovej a~kol. \parencite*{Olejarova2016}:
\begin{itemize}
    \item Metakarpofalangeálne (MCP) kĺby
    \item Proximálne interfalangeálne (PIP) kĺby
    \item Zápästia
    \item Metatarzofalangeálne (MTP) kĺby
    \item Väčšie kĺby (kolená, ramená, lakte) v~neskorších štádiách
\end{itemize}

Charakteristické sú deformity ako ulnárna deviácia prstov, labutia šija a~gombíková dierka \parencite{pavelka2018}.

\subsubsection{Mimokĺbové prejavy}

Mimokĺbové (extraartikulárne) prejavy sa vyskytujú u~približne 40~\% pacientov a~zahŕňajú:
\begin{itemize}
    \item Reumatoidné uzlíky
    \item Vaskulitídu
    \item Pľúcne postihnutie (intersticiálna pľúcna fibróza)
    \item Anémiu chronických ochorení
    \item Sekundárny Sjögrenov syndróm
    \item Kardiovaskulárne komplikácie \parencite{Rovensky2015}
\end{itemize}

\subsection{Diagnostika}

Diagnostika RA je založená na kombinácii klinických, laboratórnych a~zobrazovacích vyšetrení. Klasifikačné kritériá ACR/EULAR 2010 umožňujú včasnú diagnostiku ochorenia \parencite{aletaha2010}. Tieto kritériá nahradili pôvodné ACR kritériá z~roku 1987 a~umožňujú diagnostiku v~skorších štádiách ochorenia.

\subsubsection{Laboratórne nálezy}

Medzi kľúčové laboratórne parametre patria:
\begin{itemize}
    \item \textbf{Reumatoidný faktor (RF)} -- pozitívny u~70--80~\% pacientov
    \item \textbf{Anti-CCP protilátky} -- vyššia špecificita než RF
    \item \textbf{Zápalové parametre} -- zvýšené CRP a~ESR
    \item \textbf{Kompletný krvný obraz} -- anémia, trombocytóza \parencite{Macejova2019}
\end{itemize}

\subsubsection{Zobrazovacie metódy}

Zobrazovacie metódy zohrávajú dôležitú úlohu v~diagnostike a~sledovaní progresie RA:
\begin{itemize}
    \item \textbf{RTG vyšetrenie} -- periartikulárna osteoporóza, erózie, zúženie kĺbovej štrbiny
    \item \textbf{Ultrasonografia} -- detekcia synovitídy, power Doppler zobrazuje vaskularizáciu
    \item \textbf{Magnetická rezonancia} -- včasná detekcia kostných erózie a~edému kostnej drene \parencite{Olejarova2016}
\end{itemize}

\subsection{Hodnotenie aktivity ochorenia}

Na hodnotenie aktivity ochorenia sa používa kompozitný index DAS28 (Disease Activity Score 28), ktorý zahŕňa:
\begin{itemize}
    \item Počet bolestivých kĺbov (z 28)
    \item Počet opuchnutých kĺbov (z 28)
    \item Zápalové parametre (CRP alebo ESR)
    \item Globálne hodnotenie pacienta \parencite{smolen2018}
\end{itemize}

Podľa hodnoty DAS28 rozlišujeme:
\begin{itemize}
    \item Remisia: DAS28 $<$ 2,6
    \item Nízka aktivita: 2,6 $\leq$ DAS28 $<$ 3,2
    \item Stredná aktivita: 3,2 $\leq$ DAS28 $<$ 5,1
    \item Vysoká aktivita: DAS28 $\geq$ 5,1
\end{itemize}

\section{Kvalita života pacientov s~reumatoidnou artritídou}

Reumatoidná artritída významne ovplyvňuje kvalitu života pacientov vo všetkých jej dimenziách. Krchňavá a~kol. \parencite*{Krchnava2018} vo svojej štúdii zistili, že pacienti s~RA uvádzajú zníženú kvalitu života predovšetkým v~oblastiach fyzického fungovania, bolesti a~emocionálneho zdravia.

Bolesť predstavuje jeden z~najvýznamnejších faktorov ovplyvňujúcich kvalitu života. Je prítomná prakticky u~všetkých pacientov s~aktívnym ochorením a~limituje ich v~bežných denných aktivitách \parencite{Herbert2022}. Obmedzenie pohyblivosti kĺbov vedie k~progresívnej strate sebestačnosti a~závislosti od pomoci okolia.

\section{Liečba reumatoidnej artritídy}

Komplexná liečba RA je založená na multidisciplinárnom prístupe a~zahŕňa farmakoterapiu, rehabilitáciu a~v~indikovaných prípadoch chirurgickú liečbu \parencite{pavelka2018}.

\subsection{Farmakoterapia}

Farmakoterapia tvorí základ liečby RA. Rozlišujeme niekoľko skupín liekov:

\subsubsection{Nesteroidné antireumatiká (NSA)}
Používajú sa na symptomatickú liečbu bolesti a~zápalu, neovplyvňujú však progresiu ochorenia \parencite{Macejova2019}.

\subsubsection{Glukokortikoidy}
Majú silný protizápalový účinok, používajú sa ako premosťujúca terapia alebo pri exacerbáciách \parencite{firestein2021}.

\subsubsection{Chorobu modifikujúce lieky (DMARD)}
Základom liečby sú konvenčné syntetické DMARD, predovšetkým metotrexát, ktorý je považovaný za "zlatý štandard" v~liečbe RA. Pri nedostatočnej odpovedi sa pridávajú biologické DMARD (inhibítory TNF-$\alpha$, IL-6) alebo cielené syntetické DMARD \parencite{smolen2018}.

\section{Fyzioterapia pri reumatoidnej artritíde}

Fyzioterapia predstavuje integrálnu súčasť komplexnej liečby RA a~je neoddeliteľná od farmakoterapie \parencite{Kolar2012}. Jej význam spočíva v~udržaní alebo zlepšení funkčnej kapacity pacientov a~v~zmiernení symptómov ochorenia.

\subsection{Ciele fyzioterapie}

Podľa Koláře a~kol. \parencite*{Kolar2012} sú hlavné ciele fyzioterapie pri RA:
\begin{enumerate}
    \item Zmiernenie bolesti
    \item Udržanie alebo zlepšenie rozsahu pohybu v~kĺboch
    \item Prevencia svalovej atrofie a~udržanie svalovej sily
    \item Zlepšenie funkčnej zdatnosti a~sebestačnosti
    \item Edukácia pacienta o~ochrane kĺbov a~ergonómii
    \item Zlepšenie celkovej kvality života
\end{enumerate}

\subsection{Kinezioterapia}

Kinezioterapia je základnou súčasťou rehabilitácie pacientov s~RA. Zahŕňa pasívne, aktívne a~odporové cvičenia zamerané na udržanie pohyblivosti a~svalovej sily \parencite{Kolar2012}.

\subsubsection{Pohybová liečba v~akútnej fáze}

V~akútnej fáze ochorenia sú indikované pasívne pohyby a~polohovanie. Cieľom je prevencia kontraktúr a~udržanie rozsahu pohybu pri minimálnom zaťažení zápalovo zmenených kĺbov \parencite{Podebradsky2009}.

\subsubsection{Pohybová liečba v~subakútnej a~chronickej fáze}

V~subakútnej a~chronickej fáze sa postupne zavádzajú:
\begin{itemize}
    \item Aktívne cvičenia proti gravitácii
    \item Izometrické posilňovanie
    \item Dynamické odporové cvičenia s~nízkou záťažou
    \item Aeróbny tréning \parencite{hurkmans2009}
\end{itemize}

Zhang a~kol. \parencite*{Zhang2025} v~recentnej network meta-analýze potvrdili, že pravidelná pohybová aktivita má pozitívny vplyv na funkčný stav a~kvalitu života pacientov s~RA. Metsios a~Kitas \parencite*{metsios2020} zdôrazňujú, že cvičenie je bezpečné aj pre pacientov s~aktívnym ochorením a~nevedie k~exacerbácii príznakov.

\subsubsection{Hydrokinezioterapia}

Hydrokinezioterapia využíva vlastnosti vodného prostredia na uľahčenie pohybu a~zmiernenie bolesti. Vzťlaková sila znižuje zaťaženie kĺbov, teplo vody má relaxačný a~analgetický účinok \parencite{Kolar2012}.

\section{Elektroterapia}

Elektroterapia zahŕňa rôzne formy aplikácie elektrického prúdu na terapeutické účely. Podľa Poděbradského a~Poděbradskej \parencite*{Podebradsky2009} sa elektroterapia delí podľa frekvencie použitého prúdu na:
\begin{itemize}
    \item Nízkofrekvenčná elektroterapia (do 1000 Hz)
    \item Strednefrekvenčná elektroterapia (1000--100000 Hz)
    \item Vysokofrekvenčná elektroterapia (nad 100000 Hz)
\end{itemize}

Pri RA sa využívajú predovšetkým:
\begin{itemize}
    \item TENS (transkutánna elektrická nervová stimulácia) \parencite{brosseau2003}
    \item Interferenčné prúdy \parencite{fuentes2010}
    \item Diadynamické prúdy
    \item Galvanoterapia a~iontoforéza
\end{itemize}

\subsection{Interferenčné prúdy}

\subsubsection{Definícia a princíp}

Interferenčné prúdy (IFP) patria medzi strednofrekvenčnú elektroterapiu. Princíp spočíva v~aplikácii dvoch strednefrekvenčných prúdov (zvyčajne 4000--4100~Hz), ktoré sa v~tkanive interferujú a~vytvárajú nízkofrekvenčný modulovaný prúd \parencite{Watson2020}. Tento jav sa nazýva interferencia alebo beat fenomén.

Navrátil a~kol. \parencite*{Navratil2019} vysvetľujú, že pri interferencii dvoch prúdov s~mierne rozdielnou frekvenciou vzniká výsledný prúd s~frekvenciou zodpovedajúcou rozdielu pôvodných frekvencií. Napríklad pri frekvenciách 4000 Hz a~4100 Hz vznikne modulovaná frekvencia 100 Hz.

\subsubsection{Fyziologické účinky}

Rampazo a~Liebano \parencite*{Rampazo2022} v~prehľadovom článku sumarizovali účinky interferenčných prúdov:

\begin{enumerate}
    \item \textbf{Analgetický účinok} -- hlavný terapeutický efekt, vysvetľovaný viacerými mechanizmami:
    \begin{itemize}
        \item Teória vrátkovej kontroly bolesti (gate control theory)
        \item Stimulácia endogénnej produkcie opioidov
        \item Blokáda vedenia bolestivých vzruchov
    \end{itemize}
    
    \item \textbf{Vazodilatačný účinok} -- zlepšenie prekrvenia tkanív, podporené štúdiou Kawu a~kol. \parencite{Kawa2014}
    
    \item \textbf{Myorelaxačný účinok} -- uvoľnenie svalového napätia pri reflexných spazmoch
    
    \item \textbf{Trofotropný účinok} -- podpora regenerácie a~hojenia tkanív
\end{enumerate}

\subsubsection{Výhody interferenčných prúdov}

Hlavnou výhodou IFP je hlboký prienik do tkanív bez nepríjemných pocitov na povrchu kože, ktoré sú typické pre nízkofrekvenčné prúdy \parencite{Watson2020}. Stredné frekvencie ľahšie prekonávajú odpor kože a~umožňujú dosiahnutie vyššej intenzity v~hlbších štruktúrach \parencite{Navratil2019}.

\subsubsection{Indikácie a kontraindikácie}

\textbf{Indikácie:}
\begin{itemize}
    \item Chronické bolestivé stavy pohybového aparátu
    \item Degeneratívne ochorenia kĺbov
    \item Zápalové reumatické ochorenia (mimo akútnej fázy)
    \item Posttraumatické stavy
    \item Svalové spazmy \parencite{Podebradsky2009}
\end{itemize}

\textbf{Kontraindikácie:}
\begin{itemize}
    \item Kardiostimulátor a~iné implantované elektronické zariadenia
    \item Malignity v~oblasti aplikácie
    \item Akútne zápalové stavy
    \item Tehotenstvo (oblasť brucha a~panvy)
    \item Porušená integrita kože v~mieste aplikácie
    \item Trombóza a~tromboflebitída
    \item Kovové implantáty (relatívna kontraindikácia) \parencite{Navratil2019}
\end{itemize}

\subsubsection{Parametre aplikácie}

Podľa Watsona \parencite*{Watson2020} sa pri aplikácii IFP nastavujú nasledovné parametre:
\begin{itemize}
    \item \textbf{Nosná frekvencia:} zvyčajne 4000 Hz
    \item \textbf{Modulovaná frekvencia:} 1--100 Hz (volí sa podľa terapeutického cieľa)
    \item \textbf{Intenzita:} podprahová až nadprahová senzorická
    \item \textbf{Doba aplikácie:} 10--20 minút
    \item \textbf{Frekvencia sedení:} denne alebo obdeň
\end{itemize}

Pre analgetický účinok sa odporúča modulovaná frekvencia 80--100 Hz, pre podporu cirkulácie 1--10 Hz \parencite{Podebradsky2009}.

\subsubsection{Dôkazy o~účinnosti}

Fuentes a~kol. \parencite*{fuentes2010} v~systematickom prehľade a~meta-analýze hodnotili účinnosť IFP pri muskuloskeletálnej bolesti. Zistili, že IFP sú účinnejšie než placebo v~redukcii bolesti, avšak kvalita dôkazov bola limitovaná heterogenitou štúdií.

Kawa, Muszynska a~Kowza-Dzwonkowska \parencite*{Kawa2014} porovnávali účinnosť infračerveného žiarenia a~interferenčných prúdov u~pacientov s~degeneratívnymi ochoreniami. Obidve modality viedli k~signifikantnému zníženiu bolesti.

\section{Kombinácia elektroterapie a kinezioterapie}

Súčasné odporúčania pre fyzioterapiu pri reumatických ochoreniach zdôrazňujú dôležitosť kombinovaného prístupu \parencite{Herbert2022}. Fyzikálna terapia (vrátane elektroterapie) slúži ako príprava na aktívne cvičenie alebo ako doplnok kinezioterapie.

Gravaldi a~kol. \parencite*{Gravaldi2022} v~systematickom prehľade účinnosti fyzioterapie pri ankylozujúcej spondylitíde (ďalšom zápalovom reumatickom ochorení) potvrdili, že kombinácia rôznych fyzioterapeutických modalít je účinnejšia než monoterapia.

Predpokladáme, že kombinovaný prístup interferenčných prúdov a~kinezioterapie môže mať synergický účinok:
\begin{itemize}
    \item IFP zmiernia bolesť a~uvoľnia svalové napätie
    \item Pacient môže následne efektívnejšie cvičiť
    \item Kinezioterapia zlepší funkčný stav a~posilní svalstvo
    \item Celkový výsledok je lepší než pri použití jednotlivých modalít samostatne
\end{itemize}

\section{Hodnotenie účinnosti terapie}

\subsection{Hodnotenie bolesti}

Na hodnotenie intenzity bolesti sa najčastejšie používa vizuálna analógová škála (VAS). Pacient označí intenzitu bolesti na 10~cm úsečke, kde 0 predstavuje žiadnu bolesť a~10 najhoršiu predstaviteľnú bolesť \parencite{Herbert2022}. Alternatívne sa používa numerická škála bolesti (NRS) od 0 do 10.

\subsection{Hodnotenie rozsahu pohybu}

Rozsah pohybu (ROM -- Range of Motion) sa meria goniometrom. Hodnotí sa aktívny a~pasívny rozsah pohybu v~postihnutých kĺboch a~porovnáva sa s~fyziologickými hodnotami \parencite{Kolar2012}.

\subsection{Hodnotenie kvality života}

Na hodnotenie kvality života sa používajú štandardizované dotazníky. Medzi najpoužívanejšie patrí:
\begin{itemize}
    \item SF-36 (Short Form 36 Health Survey) -- generický dotazník
    \item HAQ (Health Assessment Questionnaire) -- špecifický pre reumatické ochorenia
    \item RAQoL (Rheumatoid Arthritis Quality of Life) -- špecifický pre RA \parencite{Krchnava2018}
\end{itemize}

\clearpage
