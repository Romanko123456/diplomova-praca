% ============================================================================
% ABSTRAKT
% Podľa metodických pokynov FZ TnUAD
% Rozsah: 100-500 slov, 3-5 kľúčových slov
% ============================================================================

\chapter*{Abstrakt}
\addcontentsline{toc}{chapter}{Abstrakt}

% Bibliografický záznam
\noindent
\textbf{BALLEK, Roman}: \textit{\nazovprace}. [Diplomová práca]. -- \univerzita. \fakulta; Vedúci diplomovej práce: \veduci. -- Trenčín: FZ TnUAD, \rok. -- XX s.

\vspace{1cm}

% Slovenský abstrakt (100-500 slov)
Diplomová práca sa zaoberá využitím fyzioterapie u~pacientov s~reumatoidnou artritídou. Hlavným cieľom práce bolo zistiť účinnosť kombinovanej liečby interferenčnými prúdmi a~kinezioterapiou v~porovnaní so samotnou kinezioterapiou. Sledovali sme vplyv terapie na intenzitu bolesti, rozsah pohybu v~postihnutých kĺboch a~subjektívne hodnotenie kvality života pacientov.

Teoretická časť práce definuje reumatoidnú artritídu, jej etiológiu, patogenézu, klinický obraz a~diagnostiku. Ďalej sa venuje možnostiam fyzioterapeutickej liečby so zameraním na interferenčné prúdy a~kinezioterapiu.

Praktická časť obsahuje výsledky výskumu realizovaného na vzorke pacientov s~potvrdenou diagnózou reumatoidnej artritídy. Pacienti boli rozdelení do experimentálnej a~kontrolnej skupiny. Údaje sme zbierali pomocou dotazníka vlastnej konštrukcie pred začiatkom a~po ukončení terapeutického cyklu. Na štatistické vyhodnotenie údajov sme použili Wilcoxonov párový test.

% TODO: Doplniť výsledky po ukončení výskumu
Výsledky ukázali... [doplniť po ukončení výskumu]

\vspace{0.5cm}

\noindent
\textbf{Kľúčové slová:} reumatoidná artritída, fyzioterapia, interferenčné prúdy, kinezioterapia, kvalita života

\clearpage

% ============================================================================
% ABSTRACT (English)
% ============================================================================

\chapter*{Abstract}
\addcontentsline{toc}{chapter}{Abstract}

% Bibliografický záznam
\noindent
\textbf{BALLEK, Roman}: \textit{\nazovpraceEN}. [Diploma thesis]. -- Alexander Dubček University of Trenčín. Faculty of Health Care; Supervisor: \veduci. -- Trenčín: FZ TnUAD, \rok. -- XX p.

\vspace{1cm}

% Anglický abstrakt (100-500 slov)
The diploma thesis deals with the use of physiotherapy in patients with rheumatoid arthritis. The main objective was to determine the effectiveness of combined treatment with interferential currents and kinesiotherapy compared to kinesiotherapy alone. We monitored the effect of therapy on pain intensity, range of motion in affected joints, and subjective assessment of patients' quality of life.

The theoretical part of the thesis defines rheumatoid arthritis, its etiology, pathogenesis, clinical presentation, and diagnosis. It also addresses physiotherapeutic treatment options with a~focus on interferential currents and kinesiotherapy.

The practical part contains the results of research conducted on a~sample of patients with a~confirmed diagnosis of rheumatoid arthritis. Patients were divided into experimental and control groups. Data were collected using a~self-constructed questionnaire before the start and after the completion of the therapeutic cycle. Wilcoxon signed-rank test was used for statistical analysis.

% TODO: Add results after research completion
Results showed... [to be completed after research]

\vspace{0.5cm}

\noindent
\textbf{Keywords:} rheumatoid arthritis, physiotherapy, interferential currents, kinesiotherapy, quality of life

\clearpage
