% ============================================================================
% Diplomová práca - Fakulta zdravotníctva TnUAD
% Autor: Roman Ballek
% Podľa metodických pokynov FZ TnUAD (5-U-021)
% ============================================================================

\documentclass[12pt,a4paper,oneside]{book}

% ============================================================================
% BALÍČKY
% ============================================================================

% Kódovanie a jazyk
\usepackage[utf8]{inputenc}
\usepackage[T1]{fontenc}
\usepackage[slovak]{babel}

% Times New Roman písmo (požiadavka TnUAD)
\usepackage{mathptmx}

% Okraje podľa TnUAD: ľavý 3,5cm, pravý 2cm, horný/dolný 2,5cm
\usepackage[left=3.5cm, right=2.0cm, top=2.5cm, bottom=2.5cm]{geometry}

% Matematika a symboly
\usepackage{amsmath}
\usepackage{amssymb}

% Grafika a obrázky
\usepackage{graphicx}
\usepackage{float}
\usepackage{caption}
\usepackage{subcaption}

% Tabuľky
\usepackage{booktabs}
\usepackage{array}
\usepackage{longtable}
\usepackage{multirow}

% Farby
\usepackage{xcolor}

% Odkazy a hyperlinky
\usepackage{hyperref}
\hypersetup{
    colorlinks=true,
    linkcolor=black,
    citecolor=black,
    urlcolor=black
}

% Bibliografia - ISO 690 podľa TnUAD
\usepackage[backend=biber, style=iso-authoryear, sorting=nyt]{biblatex}
\addbibresource{bibliografia.bib}

% Riadkovanie 1,5 (požiadavka TnUAD)
\usepackage{setspace}
\onehalfspacing

% Odsadenie odstavcov
\usepackage{indentfirst}
\setlength{\parindent}{1.25cm}

% Číslovanie strán - dole v strede (požiadavka TnUAD)
\usepackage{fancyhdr}
\fancypagestyle{plain}{
    \fancyhf{}
    \fancyfoot[C]{\thepage}
    \renewcommand{\headrulewidth}{0pt}
}
\pagestyle{plain}

% Formátovanie nadpisov podľa TnUAD
% Kapitoly: 14pt tučne, podkapitoly: 12pt tučne
\usepackage{titlesec}
\titleformat{\chapter}[display]
    {\normalfont\fontsize{14}{16}\bfseries}
    {\chaptertitlename\ \thechapter}{20pt}{\fontsize{14}{16}\bfseries}
\titleformat{\section}
    {\normalfont\fontsize{12}{14}\bfseries}{\thesection}{1em}{}
\titleformat{\subsection}
    {\normalfont\fontsize{12}{14}\bfseries}{\thesubsection}{1em}{}
\titleformat{\subsubsection}
    {\normalfont\fontsize{12}{14}\bfseries}{\thesubsubsection}{1em}{}

% Každá kapitola na novej strane
\usepackage{etoolbox}
\preto{\chapter}{\cleardoublepage}

% Obsah - nastavenia
\usepackage{tocloft}

% Prílohy
\usepackage{appendix}

% Zoznamy
\usepackage{enumitem}

% České/slovenské úvodzovky
\usepackage{csquotes}

% ============================================================================
% VLASTNÉ PRÍKAZY
% ============================================================================

% Údaje o práci
\newcommand{\nazovprace}{Využitie fyzioterapie u pacientov s reumatickým ochorením}
\newcommand{\nazovpraceEN}{Use of Physiotherapy in Patients with Rheumatic Disease}
\newcommand{\autor}{Roman Ballek}
\newcommand{\veduci}{[Meno vedúceho práce]} % DOPLNIŤ
\newcommand{\studijnyprogram}{Fyzioterapia}
\newcommand{\studijnyodbor}{Zdravotnícke vedy}
\newcommand{\fakulta}{Fakulta zdravotníctva}
\newcommand{\univerzita}{Trenčianska univerzita Alexandra Dubčeka v Trenčíne}
\newcommand{\rok}{2026}

% ============================================================================
% ZAČIATOK DOKUMENTU
% ============================================================================

\begin{document}

% ============================================================================
% ÚVODNÁ ČASŤ - bez číslovania strán (ale počítajú sa)
% ============================================================================
\frontmatter
\pagestyle{empty}

% Titulná strana
% ============================================================================
% TITULNÁ STRANA
% ============================================================================

\begin{titlepage}
    \begin{center}
        
        % Univerzita
        {\Large\bfseries \univerzita}
        
        \vspace{0.3cm}
        
        % Fakulta
        {\large \fakulta}
        
        \vspace{3cm}
        
        % Typ práce
        {\large DIPLOMOVÁ PRÁCA}
        
        \vspace{3cm}
        
        % Názov práce
        {\LARGE\bfseries \nazovprace}
        
        \vfill
        
        % Spodná časť
        \begin{flushleft}
            {\large Študijný program: \studijnyprogram}\\[0.3cm]
            {\large Študijný odbor: \studijnyodbor}\\[0.3cm]
            {\large Školiace pracovisko: \fakulta}\\[0.3cm]
            {\large Vedúci práce: \veduci}
        \end{flushleft}
        
        \vspace{1cm}
        
        \begin{flushright}
            {\large \autor}\\[0.3cm]
            {\large Trenčín \rok}
        \end{flushright}
        
    \end{center}
\end{titlepage}


% Zadanie záverečnej práce (vložiť z AIS2 ako PDF)
% \includepdf[pages=-]{zadanie.pdf}

% Čestné vyhlásenie
% ============================================================================
% ČESTNÉ VYHLÁSENIE
% ============================================================================

\chapter*{Čestné vyhlásenie}
\addcontentsline{toc}{chapter}{Čestné vyhlásenie}

\vspace{2cm}

Vyhlasujem, že som diplomovú prácu na tému \uv{\nazovprace} vypracoval samostatne s~použitím uvedenej literatúry a~na základe konzultácií s~vedúcim diplomovej práce.

\vspace{3cm}

\noindent
V Trenčíne dňa .......................

\vspace{2cm}

\hfill
\begin{tabular}{c}
    ........................................... \\
    \autor
\end{tabular}

\clearpage


% Poďakovanie (nepovinné)
% ============================================================================
% POĎAKOVANIE
% ============================================================================

\chapter*{Poďakovanie}
\addcontentsline{toc}{chapter}{Poďakovanie}

\vspace{1cm}

% TODO: Upraviť podľa potreby
Touto cestou by som sa chcel poďakovať vedúcemu diplomovej práce za odborné vedenie, cenné rady a~pripomienky, ktoré mi poskytol pri spracovaní tejto práce.

Ďakujem aj všetkým respondentom, ktorí sa zúčastnili výskumu a~ochotne vyplnili dotazníky.

V neposlednom rade ďakujem svojej rodine za podporu počas celého štúdia.

\clearpage


% Abstrakt v slovenčine a angličtine
% ============================================================================
% ABSTRAKT
% ============================================================================

\chapter*{Abstrakt}
\addcontentsline{toc}{chapter}{Abstrakt}

% Slovenský abstrakt
\textbf{BALLEK, Roman}: \textit{\nazovprace}. [Diplomová práca]. \univerzita. \fakulta; Vedúci práce: \veduci. Trenčín: TnUNI, \rok.

\vspace{0.5cm}

% TODO: Doplniť abstrakt v slovenčine
Diplomová práca sa zaoberá využitím fyzioterapie u~pacientov s~reumatoidnou artritídou. Hlavným cieľom práce je zistiť účinnosť kombinovanej liečby interferenčnými prúdmi a~kinezioterapiou v~porovnaní so samotnou kinezioterapiou. Práca sleduje vplyv terapie na intenzitu bolesti, rozsah pohybu v~postihnutých kĺboch a~subjektívne hodnotenie kvality života pacientov.

Teoretická časť práce definuje reumatoidnú artritídu, jej etiológiu, patogenézu, klinický obraz a~diagnostiku. Ďalej sa venuje možnostiam fyzioterapeutickej liečby so zameraním na interferenčné prúdy a~kinezioterapiu.

Praktická časť obsahuje výsledky výskumu realizovaného na vzorke pacientov s~potvrdenou diagnózou reumatoidnej artritídy. Pacienti boli rozdelení do experimentálnej a~kontrolnej skupiny. Údaje boli zbierané pomocou dotazníka vlastnej konštrukcie pred začiatkom a~po ukončení terapeutického cyklu.

\vspace{0.5cm}

\textbf{Kľúčové slová:} reumatoidná artritída, fyzioterapia, interferenčné prúdy, kinezioterapia, bolesť, kvalita života

\clearpage

% Anglický abstrakt
\chapter*{Abstract}
\addcontentsline{toc}{chapter}{Abstract}

\textbf{BALLEK, Roman}: \textit{\nazovpraceEN}. [Diploma thesis]. Alexander Dubček University of Trenčín. Faculty of Health Care; Supervisor: \veduci. Trenčín: TnUNI, \rok.

\vspace{0.5cm}

% TODO: Doplniť abstrakt v angličtine
The diploma thesis deals with the use of physiotherapy in patients with rheumatoid arthritis. The main objective of the thesis is to determine the effectiveness of combined treatment with interferential currents and kinesiotherapy compared to kinesiotherapy alone. The thesis monitors the effect of therapy on pain intensity, range of motion in affected joints, and subjective assessment of patients' quality of life.

The theoretical part of the thesis defines rheumatoid arthritis, its etiology, pathogenesis, clinical presentation, and diagnosis. It also addresses physiotherapeutic treatment options with a~focus on interferential currents and kinesiotherapy.

The practical part contains the results of research conducted on a~sample of patients with a~confirmed diagnosis of rheumatoid arthritis. Patients were divided into experimental and control groups. Data were collected using a~self-constructed questionnaire before the start and after the completion of the therapeutic cycle.

\vspace{0.5cm}

\textbf{Keywords:} rheumatoid arthritis, physiotherapy, interferential currents, kinesiotherapy, pain, quality of life

\clearpage


% Obsah
\tableofcontents
\clearpage

% Zoznam ilustrácií
\listoffigures
\clearpage

% Zoznam tabuliek
\listoftables
\clearpage

% Zoznam skratiek a značiek
% ============================================================================
% ZOZNAM SKRATIEK
% ============================================================================

\chapter*{Zoznam použitých skratiek}
\addcontentsline{toc}{chapter}{Zoznam použitých skratiek}

\begin{tabular}{ll}
    \textbf{ACR} & American College of Rheumatology \\
    \textbf{anti-CCP} & protilátky proti cyklickým citrulinovaným peptidom \\
    \textbf{CRP} & C-reaktívny proteín \\
    \textbf{DAS28} & Disease Activity Score 28 \\
    \textbf{DMARD} & disease-modifying antirheumatic drugs \\
    \textbf{ESR} & erytrocytová sedimentačná rýchlosť \\
    \textbf{EULAR} & European League Against Rheumatism \\
    \textbf{HAQ} & Health Assessment Questionnaire \\
    \textbf{IFP} & interferenčné prúdy \\
    \textbf{IL} & interleukín \\
    \textbf{MTP} & metatarzofalangeálny kĺb \\
    \textbf{NSA} & nesteroidné antireumatiká \\
    \textbf{PIP} & proximálny interfalangeálny kĺb \\
    \textbf{RA} & reumatoidná artritída \\
    \textbf{RF} & reumatoidný faktor \\
    \textbf{ROM} & range of motion (rozsah pohybu) \\
    \textbf{SF-36} & Short Form 36 Health Survey \\
    \textbf{TNF} & tumor nekrotizujúci faktor \\
    \textbf{VAS} & vizuálna analógová škála \\
    \textbf{WHO} & World Health Organization \\
\end{tabular}

\clearpage


% ============================================================================
% HLAVNÁ TEXTOVÁ ČASŤ - číslovanie od Úvodu
% ============================================================================

\mainmatter
\pagestyle{plain}

% Úvod (nečíslovaná kapitola, ale v obsahu)
% ============================================================================
% ÚVOD
% Podľa metodických pokynov FZ TnUAD
% Zdôvodnenie aktuálnosti a významu témy
% Písané v 1. osobe množného čísla, minulý čas (autorský plurál)
% ============================================================================

\chapter*{Úvod}
\addcontentsline{toc}{chapter}{Úvod}

% TODO: Doplniť a rozšíriť úvod
Reumatoidná artritída (RA) patrí medzi najčastejšie chronické zápalové ochorenia pohybového aparátu, ktoré výrazne ovplyvňuje kvalitu života postihnutých pacientov. Toto autoimunitné ochorenie postihuje predovšetkým synoviálne kĺby a~bez adekvátnej liečby vedie k~progresívnej deštrukcii kĺbových štruktúr, bolesti a~funkčnému obmedzeniu. Podľa epidemiologických údajov postihuje RA približne 0,5--1~\% celosvetovej populácie, pričom ženy sú postihnuté trikrát častejšie ako muži.

Komplexná liečba reumatoidnej artritídy zahŕňa farmakoterapiu, chirurgickú liečbu a~rehabilitáciu. Fyzioterapia predstavuje neoddeliteľnú súčasť liečebného procesu, pričom jej cieľom je zmierniť bolesť, zachovať alebo zlepšiť pohyblivosť kĺbov a~udržať funkčnú zdatnosť pacienta. V~posledných rokoch sa do popredia dostávajú kombinované terapeutické prístupy, ktoré integrujú rôzne fyzikálne modality s~kinezioterapiou.

Medzi moderné fyzioterapeutické metódy patrí elektroterapia, konkrétne aplikácia interferenčných prúdov. Tieto strednofrekvenčné prúdy majú preukázaný analgetický, vazodilatačný a~myorelaxačný účinok. V~kombinácii s~kinezioterapiou, ktorá zahŕňa aktívne a~pasívne cvičenia, môže byť dosiahnutý synergický efekt v~liečbe pacientov s~RA.

Napriek rozšírenému používaniu fyzioterapeutických metód v~praxi stále chýbajú dostatočné dôkazy o~účinnosti kombinovaných terapeutických protokolov u~pacientov s~reumatoidnou artritídou. Táto skutočnosť nás motivovala k~realizácii výskumu, ktorého výsledky prezentujeme v~tejto diplomovej práci.

Cieľom našej diplomovej práce bolo preskúmať účinnosť kombinovanej liečby interferenčnými prúdmi a~kinezioterapiou u~pacientov s~reumatoidnou artritídou. Práca sa zameriava na hodnotenie zmien v~intenzite bolesti, rozsahu pohybu v~postihnutých kĺboch a~subjektívnom vnímaní kvality života pacientov.

Teoretická časť práce poskytuje prehľad súčasných poznatkov o~reumatoidnej artritíde, jej diagnostike a~možnostiach fyzioterapeutickej liečby na základe analýzy domácej a~zahraničnej literatúry. Praktická časť obsahuje metodiku výskumu a~výsledky štúdie realizovanej na vzorke pacientov s~potvrdenou diagnózou RA.

\clearpage


% 1. Teoretická časť (Súčasný stav) - cca 30% práce
% ============================================================================
% TEORETICKÁ ČASŤ
% Podľa metodických pokynov FZ TnUAD - cca 30% práce
% Písané v 1. osobe množného čísla, minulý čas (autorský plurál)
% ============================================================================

\chapter{Súčasný stav riešenej problematiky}

\section{Reumatoidná artritída}

\subsection{Definícia a epidemiológia}

Reumatoidná artritída (RA) je chronické systémové autoimunitné zápalové ochorenie, ktoré primárne postihuje synoviálne kĺby. Charakterizuje sa symetrickou polyartritídou s~predilekciou malých kĺbov rúk a~nôh, pričom bez adekvátnej liečby vedie k~progresívnej deštrukcii kĺbových štruktúr \parencite{smolen2018}. Podľa Macejovej \parencite*{Macejova2019} predstavuje RA najčastejšie zápalové kĺbové ochorenie v~našej populácii.

Prevalencia RA sa pohybuje okolo 0,5--1~\% celosvetovej populácie. Ochorenie postihuje častejšie ženy než mužov v~pomere približne 3:1. Vrchol incidencie sa nachádza medzi 40.--60. rokom života, avšak RA sa môže manifestovať v~akomkoľvek veku \parencite{firestein2021}. Olejárová a~kol. \parencite*{Olejarova2016} upozorňujú, že včasná diagnostika a~liečba sú kľúčové pre prevenciu ireverzibilných kĺbových zmien.

\subsection{Etiológia a patogenéza}

Etiológia RA nie je úplne objasnená, avšak predpokladá sa multifaktoriálny pôvod zahŕňajúci genetickú predispozíciu, environmentálne faktory a~imunologickú dysreguláciu \parencite{smolen2018}.

\subsubsection{Genetické faktory}

Genetické faktory zohrávajú významnú úlohu v~rozvoji ochorenia. Najviac je asociovaná prítomnosť HLA-DR4 a~HLA-DR1 antigénov, ktoré sa nachádzajú u~60--70~\% pacientov s~RA \parencite{firestein2021}. Rovenský, Kozák a~Štvrtinová \parencite*{Rovensky2015} uvádzajú, že genetická predispozícia vysvetľuje približne 50--60~\% rizika vzniku ochorenia.

\subsubsection{Environmentálne faktory}

Medzi environmentálne faktory patria predovšetkým:
\begin{itemize}
    \item \textbf{Fajčenie} -- najvýznamnejší modifikovateľný rizikový faktor
    \item \textbf{Infekčné agens} -- predpokladá sa úloha niektorých vírusov a~baktérií
    \item \textbf{Hormonálne vplyvy} -- vysvetľujú vyššiu prevalenciu u~žien
    \item \textbf{Mikrobiom} -- aktuálne skúmaná oblasť \parencite{smolen2018}
\end{itemize}

\subsubsection{Patogenéza}

Patogenéza RA je charakterizovaná chronickým zápalom synoviálnej membrány. Aktivované T-lymfocyty, makrofágy a~fibroblasty produkujú prozápalové cytokíny (TNF-$\alpha$, IL-1, IL-6), ktoré vedú k~proliferácii synovie a~tvorbe pannu \parencite{Macejova2019}. Panus invaduje chrupavku a~subchondrálnu kosť, čo vedie k~eróziám a~deštrukcii kĺbu. Tento proces je detailne popísaný vo viacerých odborných publikáciách \parencite{pavelka2018, Olejarova2016}.

\subsection{Klinický obraz}

Klinický obraz RA je variabilný. Typický začiatok je postupný, s~rannou stuhnutosťou trvajúcou viac ako 30 minút, bolesťou a~opuchom kĺbov \parencite{Macejova2019}.

\subsubsection{Kĺbové prejavy}

Najčastejšie postihnuté kĺby sú podľa Olejárovej a~kol. \parencite*{Olejarova2016}:
\begin{itemize}
    \item Metakarpofalangeálne (MCP) kĺby
    \item Proximálne interfalangeálne (PIP) kĺby
    \item Zápästia
    \item Metatarzofalangeálne (MTP) kĺby
    \item Väčšie kĺby (kolená, ramená, lakte) v~neskorších štádiách
\end{itemize}

Charakteristické sú deformity ako ulnárna deviácia prstov, labutia šija a~gombíková dierka \parencite{pavelka2018}.

\subsubsection{Mimokĺbové prejavy}

Mimokĺbové (extraartikulárne) prejavy sa vyskytujú u~približne 40~\% pacientov a~zahŕňajú:
\begin{itemize}
    \item Reumatoidné uzlíky
    \item Vaskulitídu
    \item Pľúcne postihnutie (intersticiálna pľúcna fibróza)
    \item Anémiu chronických ochorení
    \item Sekundárny Sjögrenov syndróm
    \item Kardiovaskulárne komplikácie \parencite{Rovensky2015}
\end{itemize}

\subsection{Diagnostika}

Diagnostika RA je založená na kombinácii klinických, laboratórnych a~zobrazovacích vyšetrení. Klasifikačné kritériá ACR/EULAR 2010 umožňujú včasnú diagnostiku ochorenia \parencite{aletaha2010}. Tieto kritériá nahradili pôvodné ACR kritériá z~roku 1987 a~umožňujú diagnostiku v~skorších štádiách ochorenia.

\subsubsection{Laboratórne nálezy}

Medzi kľúčové laboratórne parametre patria:
\begin{itemize}
    \item \textbf{Reumatoidný faktor (RF)} -- pozitívny u~70--80~\% pacientov
    \item \textbf{Anti-CCP protilátky} -- vyššia špecificita než RF
    \item \textbf{Zápalové parametre} -- zvýšené CRP a~ESR
    \item \textbf{Kompletný krvný obraz} -- anémia, trombocytóza \parencite{Macejova2019}
\end{itemize}

\subsubsection{Zobrazovacie metódy}

Zobrazovacie metódy zohrávajú dôležitú úlohu v~diagnostike a~sledovaní progresie RA:
\begin{itemize}
    \item \textbf{RTG vyšetrenie} -- periartikulárna osteoporóza, erózie, zúženie kĺbovej štrbiny
    \item \textbf{Ultrasonografia} -- detekcia synovitídy, power Doppler zobrazuje vaskularizáciu
    \item \textbf{Magnetická rezonancia} -- včasná detekcia kostných erózie a~edému kostnej drene \parencite{Olejarova2016}
\end{itemize}

\subsection{Hodnotenie aktivity ochorenia}

Na hodnotenie aktivity ochorenia sa používa kompozitný index DAS28 (Disease Activity Score 28), ktorý zahŕňa:
\begin{itemize}
    \item Počet bolestivých kĺbov (z 28)
    \item Počet opuchnutých kĺbov (z 28)
    \item Zápalové parametre (CRP alebo ESR)
    \item Globálne hodnotenie pacienta \parencite{smolen2018}
\end{itemize}

Podľa hodnoty DAS28 rozlišujeme:
\begin{itemize}
    \item Remisia: DAS28 $<$ 2,6
    \item Nízka aktivita: 2,6 $\leq$ DAS28 $<$ 3,2
    \item Stredná aktivita: 3,2 $\leq$ DAS28 $<$ 5,1
    \item Vysoká aktivita: DAS28 $\geq$ 5,1
\end{itemize}

\section{Kvalita života pacientov s~reumatoidnou artritídou}

Reumatoidná artritída významne ovplyvňuje kvalitu života pacientov vo všetkých jej dimenziách. Krchňavá a~kol. \parencite*{Krchnava2018} vo svojej štúdii zistili, že pacienti s~RA uvádzajú zníženú kvalitu života predovšetkým v~oblastiach fyzického fungovania, bolesti a~emocionálneho zdravia.

Bolesť predstavuje jeden z~najvýznamnejších faktorov ovplyvňujúcich kvalitu života. Je prítomná prakticky u~všetkých pacientov s~aktívnym ochorením a~limituje ich v~bežných denných aktivitách \parencite{Herbert2022}. Obmedzenie pohyblivosti kĺbov vedie k~progresívnej strate sebestačnosti a~závislosti od pomoci okolia.

\section{Liečba reumatoidnej artritídy}

Komplexná liečba RA je založená na multidisciplinárnom prístupe a~zahŕňa farmakoterapiu, rehabilitáciu a~v~indikovaných prípadoch chirurgickú liečbu \parencite{pavelka2018}.

\subsection{Farmakoterapia}

Farmakoterapia tvorí základ liečby RA. Rozlišujeme niekoľko skupín liekov:

\subsubsection{Nesteroidné antireumatiká (NSA)}
Používajú sa na symptomatickú liečbu bolesti a~zápalu, neovplyvňujú však progresiu ochorenia \parencite{Macejova2019}.

\subsubsection{Glukokortikoidy}
Majú silný protizápalový účinok, používajú sa ako premosťujúca terapia alebo pri exacerbáciách \parencite{firestein2021}.

\subsubsection{Chorobu modifikujúce lieky (DMARD)}
Základom liečby sú konvenčné syntetické DMARD, predovšetkým metotrexát, ktorý je považovaný za "zlatý štandard" v~liečbe RA. Pri nedostatočnej odpovedi sa pridávajú biologické DMARD (inhibítory TNF-$\alpha$, IL-6) alebo cielené syntetické DMARD \parencite{smolen2018}.

\section{Fyzioterapia pri reumatoidnej artritíde}

Fyzioterapia predstavuje integrálnu súčasť komplexnej liečby RA a~je neoddeliteľná od farmakoterapie \parencite{Kolar2012}. Jej význam spočíva v~udržaní alebo zlepšení funkčnej kapacity pacientov a~v~zmiernení symptómov ochorenia.

\subsection{Ciele fyzioterapie}

Podľa Koláře a~kol. \parencite*{Kolar2012} sú hlavné ciele fyzioterapie pri RA:
\begin{enumerate}
    \item Zmiernenie bolesti
    \item Udržanie alebo zlepšenie rozsahu pohybu v~kĺboch
    \item Prevencia svalovej atrofie a~udržanie svalovej sily
    \item Zlepšenie funkčnej zdatnosti a~sebestačnosti
    \item Edukácia pacienta o~ochrane kĺbov a~ergonómii
    \item Zlepšenie celkovej kvality života
\end{enumerate}

\subsection{Kinezioterapia}

Kinezioterapia je základnou súčasťou rehabilitácie pacientov s~RA. Zahŕňa pasívne, aktívne a~odporové cvičenia zamerané na udržanie pohyblivosti a~svalovej sily \parencite{Kolar2012}.

\subsubsection{Pohybová liečba v~akútnej fáze}

V~akútnej fáze ochorenia sú indikované pasívne pohyby a~polohovanie. Cieľom je prevencia kontraktúr a~udržanie rozsahu pohybu pri minimálnom zaťažení zápalovo zmenených kĺbov \parencite{Podebradsky2009}.

\subsubsection{Pohybová liečba v~subakútnej a~chronickej fáze}

V~subakútnej a~chronickej fáze sa postupne zavádzajú:
\begin{itemize}
    \item Aktívne cvičenia proti gravitácii
    \item Izometrické posilňovanie
    \item Dynamické odporové cvičenia s~nízkou záťažou
    \item Aeróbny tréning \parencite{hurkmans2009}
\end{itemize}

Zhang a~kol. \parencite*{Zhang2025} v~recentnej network meta-analýze potvrdili, že pravidelná pohybová aktivita má pozitívny vplyv na funkčný stav a~kvalitu života pacientov s~RA. Metsios a~Kitas \parencite*{metsios2020} zdôrazňujú, že cvičenie je bezpečné aj pre pacientov s~aktívnym ochorením a~nevedie k~exacerbácii príznakov.

\subsubsection{Hydrokinezioterapia}

Hydrokinezioterapia využíva vlastnosti vodného prostredia na uľahčenie pohybu a~zmiernenie bolesti. Vzťlaková sila znižuje zaťaženie kĺbov, teplo vody má relaxačný a~analgetický účinok \parencite{Kolar2012}.

\section{Elektroterapia}

Elektroterapia zahŕňa rôzne formy aplikácie elektrického prúdu na terapeutické účely. Podľa Poděbradského a~Poděbradskej \parencite*{Podebradsky2009} sa elektroterapia delí podľa frekvencie použitého prúdu na:
\begin{itemize}
    \item Nízkofrekvenčná elektroterapia (do 1000 Hz)
    \item Strednefrekvenčná elektroterapia (1000--100000 Hz)
    \item Vysokofrekvenčná elektroterapia (nad 100000 Hz)
\end{itemize}

Pri RA sa využívajú predovšetkým:
\begin{itemize}
    \item TENS (transkutánna elektrická nervová stimulácia) \parencite{brosseau2003}
    \item Interferenčné prúdy \parencite{fuentes2010}
    \item Diadynamické prúdy
    \item Galvanoterapia a~iontoforéza
\end{itemize}

\subsection{Interferenčné prúdy}

\subsubsection{Definícia a princíp}

Interferenčné prúdy (IFP) patria medzi strednofrekvenčnú elektroterapiu. Princíp spočíva v~aplikácii dvoch strednefrekvenčných prúdov (zvyčajne 4000--4100~Hz), ktoré sa v~tkanive interferujú a~vytvárajú nízkofrekvenčný modulovaný prúd \parencite{Watson2020}. Tento jav sa nazýva interferencia alebo beat fenomén.

Navrátil a~kol. \parencite*{Navratil2019} vysvetľujú, že pri interferencii dvoch prúdov s~mierne rozdielnou frekvenciou vzniká výsledný prúd s~frekvenciou zodpovedajúcou rozdielu pôvodných frekvencií. Napríklad pri frekvenciách 4000 Hz a~4100 Hz vznikne modulovaná frekvencia 100 Hz.

\subsubsection{Fyziologické účinky}

Rampazo a~Liebano \parencite*{Rampazo2022} v~prehľadovom článku sumarizovali účinky interferenčných prúdov:

\begin{enumerate}
    \item \textbf{Analgetický účinok} -- hlavný terapeutický efekt, vysvetľovaný viacerými mechanizmami:
    \begin{itemize}
        \item Teória vrátkovej kontroly bolesti (gate control theory)
        \item Stimulácia endogénnej produkcie opioidov
        \item Blokáda vedenia bolestivých vzruchov
    \end{itemize}
    
    \item \textbf{Vazodilatačný účinok} -- zlepšenie prekrvenia tkanív, podporené štúdiou Kawu a~kol. \parencite{Kawa2014}
    
    \item \textbf{Myorelaxačný účinok} -- uvoľnenie svalového napätia pri reflexných spazmoch
    
    \item \textbf{Trofotropný účinok} -- podpora regenerácie a~hojenia tkanív
\end{enumerate}

\subsubsection{Výhody interferenčných prúdov}

Hlavnou výhodou IFP je hlboký prienik do tkanív bez nepríjemných pocitov na povrchu kože, ktoré sú typické pre nízkofrekvenčné prúdy \parencite{Watson2020}. Stredné frekvencie ľahšie prekonávajú odpor kože a~umožňujú dosiahnutie vyššej intenzity v~hlbších štruktúrach \parencite{Navratil2019}.

\subsubsection{Indikácie a kontraindikácie}

\textbf{Indikácie:}
\begin{itemize}
    \item Chronické bolestivé stavy pohybového aparátu
    \item Degeneratívne ochorenia kĺbov
    \item Zápalové reumatické ochorenia (mimo akútnej fázy)
    \item Posttraumatické stavy
    \item Svalové spazmy \parencite{Podebradsky2009}
\end{itemize}

\textbf{Kontraindikácie:}
\begin{itemize}
    \item Kardiostimulátor a~iné implantované elektronické zariadenia
    \item Malignity v~oblasti aplikácie
    \item Akútne zápalové stavy
    \item Tehotenstvo (oblasť brucha a~panvy)
    \item Porušená integrita kože v~mieste aplikácie
    \item Trombóza a~tromboflebitída
    \item Kovové implantáty (relatívna kontraindikácia) \parencite{Navratil2019}
\end{itemize}

\subsubsection{Parametre aplikácie}

Podľa Watsona \parencite*{Watson2020} sa pri aplikácii IFP nastavujú nasledovné parametre:
\begin{itemize}
    \item \textbf{Nosná frekvencia:} zvyčajne 4000 Hz
    \item \textbf{Modulovaná frekvencia:} 1--100 Hz (volí sa podľa terapeutického cieľa)
    \item \textbf{Intenzita:} podprahová až nadprahová senzorická
    \item \textbf{Doba aplikácie:} 10--20 minút
    \item \textbf{Frekvencia sedení:} denne alebo obdeň
\end{itemize}

Pre analgetický účinok sa odporúča modulovaná frekvencia 80--100 Hz, pre podporu cirkulácie 1--10 Hz \parencite{Podebradsky2009}.

\subsubsection{Dôkazy o~účinnosti}

Fuentes a~kol. \parencite*{fuentes2010} v~systematickom prehľade a~meta-analýze hodnotili účinnosť IFP pri muskuloskeletálnej bolesti. Zistili, že IFP sú účinnejšie než placebo v~redukcii bolesti, avšak kvalita dôkazov bola limitovaná heterogenitou štúdií.

Kawa, Muszynska a~Kowza-Dzwonkowska \parencite*{Kawa2014} porovnávali účinnosť infračerveného žiarenia a~interferenčných prúdov u~pacientov s~degeneratívnymi ochoreniami. Obidve modality viedli k~signifikantnému zníženiu bolesti.

\section{Kombinácia elektroterapie a kinezioterapie}

Súčasné odporúčania pre fyzioterapiu pri reumatických ochoreniach zdôrazňujú dôležitosť kombinovaného prístupu \parencite{Herbert2022}. Fyzikálna terapia (vrátane elektroterapie) slúži ako príprava na aktívne cvičenie alebo ako doplnok kinezioterapie.

Gravaldi a~kol. \parencite*{Gravaldi2022} v~systematickom prehľade účinnosti fyzioterapie pri ankylozujúcej spondylitíde (ďalšom zápalovom reumatickom ochorení) potvrdili, že kombinácia rôznych fyzioterapeutických modalít je účinnejšia než monoterapia.

Predpokladáme, že kombinovaný prístup interferenčných prúdov a~kinezioterapie môže mať synergický účinok:
\begin{itemize}
    \item IFP zmiernia bolesť a~uvoľnia svalové napätie
    \item Pacient môže následne efektívnejšie cvičiť
    \item Kinezioterapia zlepší funkčný stav a~posilní svalstvo
    \item Celkový výsledok je lepší než pri použití jednotlivých modalít samostatne
\end{itemize}

\section{Hodnotenie účinnosti terapie}

\subsection{Hodnotenie bolesti}

Na hodnotenie intenzity bolesti sa najčastejšie používa vizuálna analógová škála (VAS). Pacient označí intenzitu bolesti na 10~cm úsečke, kde 0 predstavuje žiadnu bolesť a~10 najhoršiu predstaviteľnú bolesť \parencite{Herbert2022}. Alternatívne sa používa numerická škála bolesti (NRS) od 0 do 10.

\subsection{Hodnotenie rozsahu pohybu}

Rozsah pohybu (ROM -- Range of Motion) sa meria goniometrom. Hodnotí sa aktívny a~pasívny rozsah pohybu v~postihnutých kĺboch a~porovnáva sa s~fyziologickými hodnotami \parencite{Kolar2012}.

\subsection{Hodnotenie kvality života}

Na hodnotenie kvality života sa používajú štandardizované dotazníky. Medzi najpoužívanejšie patrí:
\begin{itemize}
    \item SF-36 (Short Form 36 Health Survey) -- generický dotazník
    \item HAQ (Health Assessment Questionnaire) -- špecifický pre reumatické ochorenia
    \item RAQoL (Rheumatoid Arthritis Quality of Life) -- špecifický pre RA \parencite{Krchnava2018}
\end{itemize}

\clearpage


% 2. Cieľ práce a metodológia
% ============================================================================
% METODIKA
% ============================================================================

\chapter{Metodika práce}

\section{Charakteristika výskumného súboru}

Výskumný súbor tvorili pacienti s~potvrdenou diagnózou reumatoidnej artritídy. Podmienkou zaradenia do štúdie bola potvrdená diagnóza RA lekárom.

Predpokladaný počet respondentov: 100

Pacienti boli rozdelení do dvoch skupín:
\begin{itemize}
    \item \textbf{Experimentálna skupina} -- pacienti absolvujúci kombinovanú liečbu interferenčným prúdom a~kinezioterapiou
    \item \textbf{Kontrolná skupina} -- pacienti podstupujúci iba kinezioterapiu
\end{itemize}

\section{Metódy zberu dát}

Na zber údajov bol vytvorený dotazník vlastnej konštrukcie, ktorý pacienti vypĺňali priamo v~ambulancii pred začiatkom rehabilitačného cyklu a~po jeho ukončení.

Pred začiatkom liečby boli pacienti poučení o~správnom vyplnení dotazníkov, účele ich využitia a~o~anonymite osobných údajov.

Zber dát prebiehal od augusta 2025 do februára 2026.

\subsection{Štruktúra dotazníka}

Dotazník obsahoval:
\begin{itemize}
    \item \textbf{Demografické údaje} -- pomáhajú charakterizovať skúmanú vzorku (vek, pohlavie, dĺžka trvania ochorenia)
    \item \textbf{Intenzita bolesti} -- hodnotená pomocou VAS škály
    \item \textbf{Pohyblivosť kĺbov} -- subjektívne hodnotenie
    \item \textbf{Kvalita spánku} -- vplyv bolesti na spánok
    \item \textbf{Celková subjektívna kvalita života} -- hodnotenie pred a~po terapeutickom cykle
\end{itemize}

\section{Terapeutický protokol}

\subsection{Experimentálna skupina}

Pacienti v~experimentálnej skupine absolvovali:
\begin{enumerate}
    \item Aplikáciu interferenčných prúdov na postihnuté kĺby
    \item Kinezioterapiu zameranú na:
    \begin{itemize}
        \item Udržanie a~zlepšenie rozsahu pohybu
        \item Posilnenie svalstva
        \item Nácvik jemnej motoriky
    \end{itemize}
\end{enumerate}

\subsection{Kontrolná skupina}

Pacienti v~kontrolnej skupine absolvovali iba kinezioterapiu bez aplikácie elektroterapie.

\section{Metódy spracovania a analýzy dát}

Získané údaje boli spracované formou grafov:
\begin{itemize}
    \item \textbf{Stĺpcové grafy} -- pre jednoduchšie otázky typu áno/nie alebo kategorizáciu bolesti
    \item \textbf{Krabicové grafy} -- pre porovnanie intenzity bolesti, rozsahu pohybu a~hodnotenia kvality života pred a~po terapii
\end{itemize}

\subsection{Deskriptívna štatistika}

Pre základnú charakterizáciu vzorky bola využitá deskriptívna štatistika:
\begin{itemize}
    \item Aritmetický priemer
    \item Smerodajná odchýlka
    \item Minimum a~maximum
    \item Absolútne hodnoty
\end{itemize}

\subsection{Inferenčná štatistika}

Na vyhodnotenie štatistickej významnosti zmien bol použitý neparametrický \textbf{Wilcoxonov párový test}, vzhľadom na predpokladané nenormálne rozdelenie údajov.

Tento postup umožnil porovnať účinnosť kombinovanej liečby interferenčnými prúdmi a~kinezioterapiou s~izolovanou kinezioterapiou a~sledovať zmeny v:
\begin{itemize}
    \item Bolesti
    \item Pohyblivosti
    \item Subjektívnom hodnotení kvality života
\end{itemize}

\section{Etické aspekty výskumu}

Všetci pacienti boli informovaní o~účele výskumu a~dobrovoľne súhlasili s~účasťou. Zber dát bol anonymný a~údaje boli použité výlučne na výskumné účely.

\clearpage


% 3. Výsledky a diskusia - cca 30-40% práce
% ============================================================================
% VÝSLEDKY
% ============================================================================

\chapter{Výsledky}

% TODO: Doplniť výsledky po ukončení výskumu

\section{Charakteristika výskumného súboru}

% Príklad tabuľky - upraviť podľa skutočných údajov
\begin{table}[H]
    \centering
    \caption{Demografické charakteristiky výskumného súboru}
    \label{tab:demografia}
    \begin{tabular}{lcc}
        \toprule
        \textbf{Charakteristika} & \textbf{Experimentálna skupina} & \textbf{Kontrolná skupina} \\
        \midrule
        Počet (n) & -- & -- \\
        Vek (priemer ± SD) & -- & -- \\
        Pohlavie (ženy/muži) & -- & -- \\
        Trvanie ochorenia (roky) & -- & -- \\
        \bottomrule
    \end{tabular}
\end{table}

\section{Hodnotenie intenzity bolesti}

% TODO: Doplniť výsledky a grafy

\subsection{Intenzita bolesti pred terapiou}

% Sem vložiť graf alebo tabuľku

\subsection{Intenzita bolesti po terapii}

% Sem vložiť graf alebo tabuľku

\subsection{Porovnanie zmeny intenzity bolesti medzi skupinami}

% Sem vložiť krabicový graf

\section{Hodnotenie rozsahu pohybu}

% TODO: Doplniť výsledky a grafy

\subsection{Rozsah pohybu pred terapiou}

% Sem vložiť graf alebo tabuľku

\subsection{Rozsah pohybu po terapii}

% Sem vložiť graf alebo tabuľku

\subsection{Porovnanie zmeny rozsahu pohybu medzi skupinami}

% Sem vložiť krabicový graf

\section{Hodnotenie kvality života}

% TODO: Doplniť výsledky a grafy

\subsection{Kvalita života pred terapiou}

% Sem vložiť graf alebo tabuľku

\subsection{Kvalita života po terapii}

% Sem vložiť graf alebo tabuľku

\subsection{Porovnanie zmeny kvality života medzi skupinami}

% Sem vložiť krabicový graf

\section{Štatistická analýza}

% TODO: Doplniť výsledky štatistických testov

\subsection{Testovanie hypotézy H1}

% Výsledky Wilcoxonovho testu pre bolesť

\subsection{Testovanie hypotézy H2}

% Výsledky Wilcoxonovho testu pre rozsah pohybu

\subsection{Testovanie hypotézy H3}

% Výsledky Wilcoxonovho testu pre kvalitu života

\clearpage

% ============================================================================
% DISKUSIA
% ============================================================================

\chapter{Diskusia}

% TODO: Doplniť diskusiu po ukončení výskumu

V~tejto kapitole budú interpretované získané výsledky v~kontexte stanovených hypotéz a~porovnané s~výsledkami podobných štúdií publikovaných v~odbornej literatúre.

\section{Interpretácia výsledkov}

% TODO: Doplniť interpretáciu

\subsection{Vplyv kombinovanej terapie na intenzitu bolesti}

% Diskusia k H1

\subsection{Vplyv kombinovanej terapie na rozsah pohybu}

% Diskusia k H2

\subsection{Vplyv kombinovanej terapie na kvalitu života}

% Diskusia k H3

\section{Porovnanie s inými štúdiami}

% TODO: Porovnať výsledky s literatúrou

\section{Limity štúdie}

% TODO: Doplniť limity
Medzi limity štúdie môžeme zaradiť:
\begin{itemize}
    \item Veľkosť výskumného súboru
    \item Subjektívnosť niektorých hodnotených parametrov
    \item Nemožnosť zaslepenia terapie
\end{itemize}

\section{Odporúčania pre prax}

% TODO: Doplniť odporúčania

\clearpage


% Záver (nečíslovaná kapitola)
% ============================================================================
% ZÁVER
% Podľa metodických pokynov FZ TnUAD
% Písané v 1. osobe množného čísla, minulý čas (autorský plurál)
% ============================================================================

\chapter*{Záver}
\addcontentsline{toc}{chapter}{Záver}

Diplomová práca sa zaoberala využitím fyzioterapie u~pacientov s~reumatoidnou artritídou so zameraním na hodnotenie účinnosti kombinovanej liečby interferenčnými prúdmi a~kinezioterapiou.

V~teoretickej časti práce sme na základe analýzy domácej a~zahraničnej literatúry definovali reumatoidnú artritídu ako chronické autoimunitné zápalové ochorenie postihujúce synoviálne kĺby. Predstavili sme súčasné poznatky o~etiológii, patogenéze, klinickom obraze a~diagnostike tohto ochorenia. Podrobne sme sa venovali možnostiam fyzioterapeutickej liečby so zameraním na interferenčné prúdy a~kinezioterapiu.

% TODO: Doplniť po ukončení výskumu - zhrnutie hlavných zistení

\textbf{K~hlavnému cieľu:}

Hlavným cieľom práce bolo zistiť, ako kombinovaná liečba interferenčnými prúdmi a~kinezioterapiou ovplyvňuje intenzitu bolesti, rozsah pohybu a~subjektívne hodnotenie kvality života pacientov s~reumatoidnou artritídou.

% [Doplniť zhrnutie hlavných zistení]

\textbf{K~hypotéze H1:}

Predpokladali sme, že pacienti s~RA, ktorí absolvujú terapiu interferenčnými prúdmi v~kombinácii s~kinezioterapiou, uvádzajú nižšiu intenzitu bolesti než pacienti liečení len kinezioterapiou.

% [Doplniť: Hypotéza bola/nebola potvrdená...]

\textbf{K~hypotéze H2:}

Predpokladali sme, že kombinácia interferenčných prúdov a~kinezioterapie má väčší pozitívny vplyv na rozsah pohybu v~postihnutých kĺboch ako samotná kinezioterapia.

% [Doplniť: Hypotéza bola/nebola potvrdená...]

\textbf{K~hypotéze H3:}

Predpokladali sme, že pacienti liečení kombinovanou terapiou subjektívne hodnotia kvalitu svojho života pozitívnejšie ako pacienti v~skupine bez elektroterapie.

% [Doplniť: Hypotéza bola/nebola potvrdená...]

\vspace{1cm}

Výsledky našej štúdie poukazujú na potenciálny prínos kombinovanej terapie v~rehabilitácii pacientov s~reumatoidnou artritídou. Fyzioterapia predstavuje neoddeliteľnú súčasť komplexnej liečby RA a~má nezastupiteľnú úlohu v~zlepšovaní funkčného stavu a~kvality života pacientov.

Pre klinickú prax odporúčame zaradenie interferenčných prúdov do terapeutického protokolu ako prípravnú fázu pred kinezioterapiou. Tento prístup môže zvýšiť účinnosť rehabilitácie a~zlepšiť toleranciu pohybovej liečby u~pacientov s~bolestivými stavmi.

Ďalší výskum by mal byť zameraný na dlhodobé sledovanie účinnosti kombinovanej terapie a~na optimalizáciu parametrov aplikácie interferenčných prúdov u~pacientov s~reumatickými ochoreniami.

\clearpage


% ============================================================================
% ZÁVEREČNÁ ČASŤ
% ============================================================================

\backmatter

% Zoznam literatúry (ISO 690, abecedne)
\printbibliography[title={Zoznam bibliografických odkazov}]

% Deklarácia použitia AI (ak bola použitá)
% ============================================================================
% DEKLARÁCIA POUŽITIA UMELEJ INTELIGENCIE
% Podľa metodických pokynov TnUAD
% ============================================================================

\chapter*{Deklarácia použitia umelej inteligencie}
\addcontentsline{toc}{chapter}{Deklarácia použitia umelej inteligencie}

% Ak ste NEPOUŽILI AI, zakomentujte nasledujúce a odkomentujte text nižšie
% ----------------------------------------------------------------------------

Pri vypracovaní tejto diplomovej práce boli použité nasledovné nástroje generatívnej umelej inteligencie:

\begin{table}[H]
    \centering
    \begin{tabular}{|p{4cm}|p{3cm}|p{6cm}|}
        \hline
        \textbf{Názov nástroja} & \textbf{Dátum použitia} & \textbf{Účel použitia} \\
        \hline
        % Príklad - upraviť podľa skutočnosti:
        ChatGPT & december 2025 & Konzultácie pri štruktúrovaní teoretickej časti \\
        \hline
        % Pridať ďalšie riadky podľa potreby
    \end{tabular}
\end{table}

Autor potvrdzuje, že všetky výstupy generatívnej AI boli kriticky zhodnotené a upravené, pričom autor preberá plnú zodpovednosť za obsah práce.

% ----------------------------------------------------------------------------
% Ak ste AI NEPOUŽILI, odkomentujte nasledovné:
% ----------------------------------------------------------------------------
% Vyhlasujem, že pri vypracovaní tejto diplomovej práce som nepoužil žiadne
% nástroje generatívnej umelej inteligencie.

\clearpage


% Prílohy (nezapočítavajú sa do rozsahu)
\appendix
% ============================================================================
% PRÍLOHY
% ============================================================================

\chapter{Prílohy}

\section*{Príloha A: Dotazník vlastnej konštrukcie}
\addcontentsline{toc}{section}{Príloha A: Dotazník vlastnej konštrukcie}

% TODO: Vložiť dotazník

\clearpage

\section*{Príloha B: Informovaný súhlas pacienta}
\addcontentsline{toc}{section}{Príloha B: Informovaný súhlas pacienta}

% TODO: Vložiť informovaný súhlas

\clearpage


\end{document}
